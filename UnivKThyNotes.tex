\documentclass[letterpaper]{article}
\usepackage{fancyhdr}
\usepackage[margin=1in]{geometry}
\usepackage{amsmath,amsthm,amssymb,amsfonts,amsopn,mathrsfs,mathtools}
%\usepackage{parskip}
\usepackage{tikz}
\usetikzlibrary{arrows}
\usepackage[all]{xy}
\usepackage{pdfpages}
\usepackage[compact]{titlesec}
\usepackage[normalem]{ulem}
\usepackage{cite}
\usepackage{hyperref}
\usepackage{color}
\usepackage{enumitem}
\usepackage{tabulary}
\usepackage{titlesec}
\usepackage{hyperref}

\hypersetup{colorlinks=true,linkcolor=blue}

\newtheorem{lemma}{Lemma}
\newtheorem{theorem}{Theorem}
\newtheorem{proposition}[lemma]{Proposition}
\theoremstyle{definition}
\newtheorem{corollary}[lemma]{Corollary}
\newtheorem{example}[lemma]{Example}
\newtheorem{definition}[lemma]{Definition}

\newtheorem{remark}[lemma]{Remark}
\newtheorem{question}{Question}
\newtheorem{conjecture}{Conjecture}
\newtheorem{construction}{Construction}



\newcommand{\Z}{\mathbb Z}

\newcommand{\mbb}{\mathbb}
\newcommand{\mc}{\mathcal}


\DeclareMathOperator{\Spec}{Spec}
\DeclareMathOperator*{\holim}{holim}
\DeclareMathOperator*{\hocolim}{hocolim}
\DeclareMathOperator*{\colim}{colim}
\DeclareMathOperator*{\im}{im}
\DeclareMathOperator*{\coker}{coker}

\setcounter{secnumdepth}{4}
\titleformat{\paragraph}
{\normalfont\normalsize\bfseries}{\theparagraph}{1em}{}
\titlespacing*{\paragraph}
{0pt}{3.25ex plus 1ex minus .2ex}{1.5ex plus .2ex}





 \setlength{\headheight}{0pt}
  \lhead{Daniel, Brian, Tsutomu}\chead{UnivKThy Notes}\rhead{Summer 2017}
  \lfoot{}\cfoot{\thepage}\rfoot{}
  \pagestyle{fancy} 


\begin{document}
I'm using the Arxiv version 5 Feb 2013

\section{Spectral Categories}

\subsection{Summary}
\subsubsection{Important definitions}
\begin{tabular}{|c|c|}
\hline
Definition & Page Number\\
\hline
\hline
Spectral category & 8\\
\hline
$DK$ equivalence &9\\
\hline
Module over a spectral category &10\\
\hline
Triangulated closure &10 \\
\hline
Triangulated equivalence  & 11\\
\hline
Morita equivalence & 11\\
\hline
\end{tabular}

\subsubsection{Notation}
\begin{itemize}
\item $Cat_S$ is the category of small spectral categories and
  spectral functors. 
\item $Cat_T$ is the category of small simplicial categories and
  simplicial functors. 
\end{itemize}
\subsubsection{Key results}
\begin{enumerate}
\item There's a Quillen adjunction
\[
\xymatrix{Cat_S \ar@<.5ex>[r]^{\Sigma^\infty_+} & \ar@<.5ex>[l]^{\Omega^\infty} Cat_T}
\]

where $\Omega^\infty(F(A,B))$ is the zeroth space functor or
equivalently the simplicial
set $[n] \mapsto Hom_{Spt}(\mbb S \otimes \Delta^n,F(A,B)) \cong
Hom_{Spc}(\Delta^n,\Omega^\infty F(A,B)) \cong \Omega^\infty
F(A,B)_n$. This is the main theorem of Tabuada's paper ``Homotopy
Theory of Spectral Categories'' (see references). We can modify
$Cat_S$ up to weak equivalence to make this a simplicial Quillen
adjunction.

\begin{remark}
It might also be helpful to recall that, given a monoidal functor from
a monoidal category $M$ to a monoidal category $N$, any category
enriched over $M$ can be reinterpreted as a category enriched over
$N$. Furthermore, we recover the underlying category of an enriched
category by considering the functor $M(I,-) : M \rightarrow Set$ where
$I$ is the unit of the monoidal structure.
\end{remark}

\item There's a model category $\widehat{\mc A}$ of spectral $\mc
  A$-modules (these are defined as functors, so we put the projective
  model structure on the functor category). The referenced paper here
  is \href{/References/StabCatAreModuleCat.pdf}{Stable model categories are categories of modules} by Schwede
  and Shipley.
\end{enumerate}

\subsection{Clarifying Remarks}

\begin{itemize}
\item The definition of $A$-module is a generalization of the ordinary
  one. Recall that a ring is equivalently a preadditive category (an
  $Ab$-enriched category) with
  one object, and a module is a functor from this ring to abelian
  groups. Along the same vein, a ring spectrum is a spectral category with
  one object, and a module over this ring spectrum is a functor from
  this category to spectra.
\item A category of enriched functors is itself enriched. Recall that
  $Nat(F,G) = \int_C Hom(F-,G-)$. Via the theory of enriched ends we
  can define a mapping spectrum to be $\int_C A(F-,G-)$, where $A$
  denotes the mapping spectrum of two objects in our enriched
  category.
\item Given a functor $F : \mc A \rightarrow \mc B$, we get an obvious
  restriction functor $F^* : mod B := \widehat B \rightarrow \widehat A$
  given by precomposition with $F$ (recall the definition of a module
  as a functor!). It has a left adjoint functor $F_!$  given by an enriched
  coend. It sends an $A$-module $N$ to the coequalizer of
\[
\xymatrix{\bigvee_{o,p \in A} N(p) \wedge A(o,p) \wedge B(-,{F(o)})
  \ar@<.5ex>[r] \ar@<-.5ex>[r] & \bigvee_{o \in A} N(o) \wedge B(-,F(o))}.
\]

Recall that representable objects are ``free on one
generator''. Indeed, the usual definition of a free object is that
maps out of it are determined by maps out of a basis. We know, by the
Yoneda lemma, that maps out of representable objects are determined by
maps out of the identity morphism.

\item The paper omits the definition of a pretriangulated spectral
  category. It's in the
  \href{References/LocalizationTHH.pdf}{Mandell-Blumberg} paper in the
  references, definition 5.4.
\end{itemize}

\subsection{Background Material}

While the theory of infinity categories allows us to work
``coordinate-free'' in some sense, we first need some examples of
meaningful infinity categories and functors between them. One way to
get a slew of presentable $\infty$-categories is to take the underlying
infinity category of a left proper, simplicial, combinatorial model
category. We'll define the simplicial nerve after introducing
combinatorial model categories.

\subsubsection{Combinatorial Model Categories}
\begin{definition}
An infinite cardinal $\kappa$ is a \emph{regular cardinal} if it
satisfies the following property, which we think of as requiring the
collection of sets of smaller cardinality to be ``closed
under union'':
\begin{itemize}
\item no set of cardinality $\kappa$ is the union of fewer than
  $\kappa$ sets of cardinality less than $\kappa$.
\end{itemize}
\end{definition}

\begin{example}
$\aleph_0$, or the cardinality of $\mbb N$, is a regular
cardinal. This is because any set of cardinality less than $\aleph_0$
is finite, and no infinite set is a finite union of finite sets.
\end{example}

\begin{example}
Any successor cardinal is regular. For $\aleph_1$, this follows from
the fact that a countable union of countable sets is countable (we
need choice here).
\end{example}



\begin{definition}
Let $\kappa$ be an infinite regular cardinal. Then a
\emph{$\kappa$-filtered category} is one such that any diagram $F :D
\rightarrow C$, where $D$ has fewer than $\kappa$ morphisms admits an
extension $\widetilde F: D^+ \rightarrow C$ (i.e. $F$ has a
cocone). Here $D^+$ is the category obtained by freely adjoining a
terminal object to $D$.
\end{definition}

\begin{example}
Let $\kappa = \omega$ (or $\aleph_0$ in the notation we've been
using). Then we're requiring every FINITE diagram in $C$ to have a
cocone. This is equivalent to the usual definition of a filtered
category: a nonempty category s.t. each pair of objects has a join,
and for any parallel morphisms $f,g: c_1 \rightarrow c_2$ in $C$ there
exists a morphism $h : c_2 \rightarrow c_3$ such that $hf = hg$. In
other words, we can build a cocone for any finite diagram from these cocones.
\end{example}

\begin{example}
A preorder (there exists a unique morphism between any two objects) is
$\omega$-filtered precisely when it is \emph{directed}, i.e. any two objects
have a join.
\end{example}

\begin{definition}
Let $\kappa$ be a regular cardinal. Then an object $X$ such that
$C(X,-)$ commutes with $\kappa$-filtered colimits is called \emph{$\kappa$-compact}.
\end{definition}

\begin{definition}
An object $X$ of a category is \emph{small} if it is $\kappa$-compact
for some regular cardinal $\kappa$.
\end{definition}

\begin{definition}
A category $C$ is \emph{locally presentable} if 
\begin{enumerate}
\item $C$ is a locally small category
\item $C$ has all small colimits
\item there exists a small set $S\hookrightarrow Obj(C)$ of
  $\lambda$-small objects that generates $C$ under $\lambda$-filtered
  colimits.
\item every obect in $C$ is a small object.
\end{enumerate}
\end{definition}

\begin{definition}
Let $C$ be a category and $I \subset Mor(C)$. Let \emph{$cell(I)$} be
  the class of morphisms obtained by transfinite composition of
  pushouts of coproducts of elements in $I$.
\end{definition}

\begin{definition}
A model category $C$ is \emph{cofibrantly generated} if there are
small sets of morphisms $I,J \subset Mor(C)$ such that
\begin{itemize}
\item $cof(I)$, the set of retracts of elements in $cell(I)$, is
  precisely the collection of cofibrations of $C$.
\item $cof(J)$ is precisely the collection of acyclic cofibrations in
  $C$; and
\item $I$ and $J$ permit the small object argument. 
\end{itemize}
\end{definition}

\begin{definition}(Smith)
A model category $C$ is \emph{combinatorial} if it is

\begin{itemize}
\item locally presentable as a category, and 
\item cofibrantly generated as a model category.
\end{itemize}


\end{definition}

\begin{example}
The category $sSet$ with the standand model structure on simplicial
sets is a combinatorial model category. 
\end{example}

\begin{example}
The category $sSet$ with the Joyal model structure (so that the
quasi-categories are the fibrant objects) is combinatorial.
\end{example}

\begin{question}
Why should we care that a model category is combinatorial?
\end{question}




We'll define a left-proper model category, then give a fundamental
result of Dugger's:

\begin{definition}
A model category is \emph{left proper} if weak equivalence is
preserved by pushout along cofibrations.
\end{definition}

\begin{example}
A model category in which all objects are cofibrant is left
proper. This includes the standard model structure on simplicial sets,
as well the injective model structure on simplicial presheaves. This
follows from the Reedy lemma, which allows us to calculate homotopy
pushouts by considering diagrams s.t. the objects are cofibrant and
one of the maps is a cofibration (the point is that replacing this
diagram with a cofibrant diagram in the projective model structure on
diagrams is an acyclic cofibration of diagrams, not just a weak equivalence).
\end{example}

\begin{theorem}(Dugger)

Every combinatorial model category is Quillen equivalent to a left
proper simplicial combinatorial model category. 
\end{theorem}


Now, we have the following extremely useful result on Bousfield
localizations:


\begin{theorem}
If $C$ is a left proper, simplicial, combinatorial model category, and
$S \subset Mor(C)$ is a small set of morphisms, then the left
Bousfield localization $L_SC$ does exist as a combinatorial model
category. Moreover, the fibrant objects of $L_SC$ are precisely the
$S$-local objects, and $L_S$ is left proper and simplicial.
\end{theorem}

In the context of infinity categories, we have some crucial
results of Lurie. For these to make sense, however, we need to
introduce the simplicial nerve construction. This is the
generalization of the ordinary nerve construction to simplicially
enriched categories.

\begin{definition}
We'll define a cosimplicial simplcially enriched category $S$. The
objects of $S[n]$ are $\{0,1,\dots,n\}$; the hom objects $S[n]_{i,j}
\in sSet$ for $i,j \in \{0,1,\dots,n\}$ are the nerves
\[
S[n](i,j) = N(P_{i,j})
\]
of the poset $P_{i,j}$ which is the poset of subsets of $[i,j]$ that
contain both $i$ and $j$ with partial order given by inclusion.
\end{definition}

\begin{definition}
The simplicial nerve of a simplicial category is the simplicial set
characterized by 
\[
Hom_{sSet}(\Delta[n],N(C)) = Hom_{SSetCat}(S[n],C).
\]
\end{definition}

\begin{theorem}(HTT A.3.7.6)
Let $C$ be an $\infty$-category. The following conditions are
equivalent:

\begin{enumerate}
\item The $\infty$-category $C$ is presentable.
\item There exists a combinatorial simplicial model category $A$ and
  an equivalence $C \cong N(A^\circ)$.
\end{enumerate}

Here $A^{\circ}$ is the underlying category of bifibrant objects.
\end{theorem}

\begin{remark}(HTT A.3.7.7)

Let $A$ and $B$ be combinatorial simplicial model categories. Then the
underlying $\infty$-categories $N(A^\circ)$ and $N(B^\circ)$ are
equivalent iff $A$ and $B$ can be joined by a chain of simplicial
Quillen equivalences. 
\end{remark}

\begin{theorem}(HTT 5.2.4.6)
Let $A$ and $B$ be simplicial model categories, and let 
\[
\xymatrix{A \ar@<.5ex>[r]^F & \ar@<.5ex>[l]^G B}
\]

be a simplicial Quillen adjunction. This descends to an adjunction on
the underlying $\infty$-categories.
\end{theorem}

\section{Stable $\infty$-categories}

\subsection{Summary}
\subsubsection{Important definitions}
\begin{tabular}{|c|c|}
\hline
Definition & Page Number\\
\hline
\hline
Stable $\infty$ category & 14\\
\hline
Idempotent complete & 14\\
\hline
Spectral Yoneda embedding & 15\\
\hline
Stably representable functor & 16\\
\hline
Accessible $\infty$-category & 17\\
\hline
Presentable $\infty$-category & 17\\
\hline
\end{tabular}

\subsubsection{Notation}

\begin{itemize}
\item $Cat_\infty$ is the $\infty$-category of small $\infty$ categories.
\item $Cat^{ex}_\infty$ is the $\infty$-category of small stable $\infty$-categories and exact functors. 
\item $Cat^{perf}_\infty$ is the $\infty$-category of small idempotent complete stable $\infty$-categories (and exact functors).
\end{itemize}

\subsubsection{Key results}

\begin{enumerate}
\item If $C$ is a pretriangulated spectral category, then $\pi_0 C$ admits a triangulated category structure.

\item If $C$ is a stable $\infty$-category, then $ho C$ admits a triangulated category structure. 

\item The inclusion $Cat^{perf}_\infty \rightarrow Cat^{ex}_\infty$ has a left adjoint called idempotent completion.

\item If $C$ is a $\infty$-category with finite limits, there is a stable $\infty$-category $Stab(C)$ with a limit preserving functor $\Omega^\infty : Stab(C) \rightarrow C$. In particular, this is accomplished by taking $Stab(C) = Sp(C)$, the category of spectrum objects of $C$. If $C$ is already stable, then $\Omega^\infty$ is an equivalence, with inverse $\Sigma^\infty : C \rightarrow Sp(C)$. 

\item If $C$ is a stable $\infty$-category, there is a \textit{spectral Yoneda embedding}
$$Y : C \simeq Sp(C_*) \rightarrow Sp(Fun(C^{op}, N(T)^{cf})_*) \simeq Fun(C^{op}, S_\infty)$$
with an adjoint \textit{mapping spectrum functor}
$$Map : C^{op} \times C \rightarrow S_\infty$$

\item If $C$ is a stable $\infty$-category, a $C^{op} \rightarrow S_\infty$ is stably representable if and only if it is represented by the suspension spectrum $\Sigma^\infty z$ of some $z \in C$, and this object is unique upto equivalence. 

\end{enumerate}

\subsection{Clarifying Remarks}

\begin{enumerate}
\item Idempotent completion is unique upto equivalence by HTT Proposition 5.1.4.9. Existence is obtained by taking the full subcategory of $Pre(C)$ spanned by objects that are retracts of objects in the image of $C$ under the Yoneda embedding.
\item For this paper we assume stable $\infty$-categories to be pointed, i.e. they have zero objects. In the description of spectral Yoneda embedding above, $*$ indicates the subcategory of pointed objects. 
\end{enumerate}

\subsection{Background Material}

\subsubsection{Pretriangulated Spectral Categories}

The material here is from Section 5 of \href{References/LocalizationTHH.pdf}{Mandell-Blumberg} (in the higher K-theory paper, they say Section 4 but that's either a typo or just not updated). 

\begin{definition}
A spectral category $C$ is \textit{pretriangulated} if 
\begin{enumerate}
\item There exists an object 0 of $C$ such that $C(-,0)$ is homotopically trivial. This means that it is weakly equivalent to the constant functor $*$ at the one point symmetric spectrum. 
\item Whenever a $C$-module $M$ has the property that $\Sigma M$ is weakly equivalent to a representable $C$-module $C(-,c)$, then $M$ is weakly equivalent to some representable $C$-module $C(-,d)$.
\item Whenever the $C$-modules $M$ and $N$ are weakly equivalent to representables $C(-,a)$ and $C(-,b)$ respectively, the homotopy cofiber of any map of $C$-modules $M \rightarrow N$ is weakly equivalent to a representable $C$-module.
\end{enumerate}
\end{definition}

\begin{remark}
The first condition says that the homotopy category $\pi_0 C$ has a zero object. The second condition gives a desuspension functor on $\pi_0 C$ and the third condition gives a suspension functor on $\pi_0 C$. 
\end{remark}

\begin{definition}
A spectral functor between spectral categories $F : C \rightarrow D$ is a \textit{DK-embedding} if for all objects $a,b$ in $C$, the induced map of spectra
$C(a,b) \rightarrow D(Fa,Fb)$ is a weak equivalence. 
\end{definition}

\begin{remark}
In Blumberg-Mandell, a DK equivalence is a DK embedding satisfying one of the following equivalent conditions 
\begin{enumerate}
\item For all object $d$ of $D$, there is an object $c$ of $C$ such that $D(-,d)$ and $D(-,Fc)$ are naturally isomorphic as $D$-modules.
\item The induced functor on the ``graded homotopy categorie'' $\pi_* C \rightarrow \pi_* D$ is an equivalence of categories. 
\end{enumerate}
In Blumberg-Gepner-Tabuada, this second condition is relaxed to $\pi_0 C \rightarrow \pi_0 D$ being an equivalence of categories. This is because a spectral functor between pretriangulated spectral categories is a DK equivalence if and only if it is a DK embedding and the induced functor on $\pi_0$ is an equivalence of categories. 
\end{remark}

\begin{theorem}(Blumberg-Mandell, Theorem 5.5)
Any small spectral category $C$ DK-embeds into a small pretriangulated spectral category. 
\end{theorem}

\begin{remark}
We should think of this as taking the closure of $C$ under cofibration sequences and desuspensions in $C$-modules, via the Yoneda embedding. Note that this can be made functorial and it gives the ``minimal pretriangulated closure" of $C$. 
\end{remark}

\begin{definition}
A \textit{four-term Puppe sequence} in $\pi_0 C$ is a sequence of the form
$$a \rightarrow b \rightarrow c \rightarrow \Sigma a$$
if there exist a map of $C$-modules $f : M \rightarrow N$ such that the sequence
$$M \rightarrow N \rightarrow Cf \rightarrow \Sigma M$$
is isomorphic to the above sequence in the derived category of $C$-modules via the Yoneda embedding, and further the equvalence $\Sigma M \simeq C(-,\Sigma a) \simeq \Sigma C(-,a)$ is the suspension of the isomorphism $M \simeq C(-,a)$. 
\end{definition}

\begin{theorem}(Blumberg-Mandell, Theorem 5.6)
Given a pretriangulated spectral category $C$, its homotopy category $\pi_0 C$ is triangulated with distinguished triangles the above four-term Puppe sequences. 
\end{theorem}

\begin{proof}
Proof of Theorem 5.6 is just observing that $\pi_0 C$ embeds as a full subcategory of the homotopy category of $C$-modules (with projective model structure) and checking that it is closed under suspensions, desuspensions and distinguished triangles. 
\end{proof}

\subsubsection{$\infty$-Categories}

\begin{definition} The \textbf{Joyal model structure} on simplicial sets is defined as follows : 
\begin{itemize}
\item Cofibrations are levelwise monomorphisms.
\item Weak equivalences are \textit{(weak) categorical equivalence}. These are maps $f : A \rightarrow B$ such that for any $\infty$-category $X$, the map $X^B \rightarrow X^A$ induces an isomorphism on the fundamental category (the homotopy category) $ho C$. 
\item Fibrations are determined by the above. 
\end{itemize}
\end{definition}

\begin{remark}
All objects are cofibrant, and the fibrant objects are precisely $\infty$-categories. 
\end{remark}

\begin{theorem}(HA Theorem 1.1.2.15) Let $C$ be a stable $\infty$-category. Then $ho C$ has a triangulated category structure with distinguished triangles coming from cofiber sequences. 
\end{theorem}

\begin{remark}
There is a correspondance between pretriangulated spectral categories and stable $\infty$-categories. For example, if $C$ is a pretriangulated spectral category, then $N((ModC)^{cf})$ is a stable $\infty$-category. 
\end{remark}

\subsubsection{Idempotent completion}

HTT 4.4.5 gives a good overview on the definition of idempotent completion for $\infty$-categories, while comparing it with the classical notion. 

\begin{remark} The notion of retracts between classical and $\infty$-categorical settings are a bit different. An ordinary category $X$ is said to be \textit{idempotent complete} if every idempotent map $X \rightarrow X$ comes from some retract $Y$ of $X$. In such a situation $Y$ can be determined uniquely as an equalizer (or a coequalizer). Hence, if $C$ has finite limits or finite colimits, then $C$ is idempotent complete. 

This is not the case for $\infty$-categories. Consider the category $C_*(R)$ consisting of bounded chain complex of finite rank free $R$-modules and consider $N(C_*(R))$, which is actuallt a stable $\infty$-category. Hence it admits finite limits and colimits, but it is idempotent complete if and only if every finitely generated projective $R$-module is stably free. 

The problem is that an idempotent in an $\infty$-category shouldn't be just a morphism $e$ with $e \circ e \simeq e$ in $ho C$. It should specify homotopies on how to relate multiple compositions $e \circ e \circ ... \circ e \simeq e$. To achieve this, in HTT 4.4.5, simplicial sets called $Idem^+, Idem$ and $Ret$ are introduced. Now idempotents, weak retractions, strong retractions in $C$ are respectively defined to be maps of simplicial sets from $Idem, Ret, Idem^+$ to $C$. $C$ is \textit{idempotent complete} if every idempotent $F : Idem \rightarrow C$ has a colimit. This can be shown to be equivalent to the definition in the paper using results in HTT 5.1.4 and 5.1.5.
\end{remark}

\end{document}