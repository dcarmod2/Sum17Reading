\documentclass[letterpaper]{article}
\usepackage{fancyhdr}
\usepackage[margin=1in]{geometry}
\usepackage{amsmath,amsthm,amssymb,amsfonts,amsopn,mathrsfs,mathtools}
%\usepackage{parskip}
\usepackage{tikz}
\usetikzlibrary{arrows}
\usepackage[all]{xy}
\usepackage{pdfpages}
\usepackage[compact]{titlesec}
\usepackage[normalem]{ulem}
\usepackage{cite}
\usepackage{hyperref}
\usepackage{color}
\usepackage{enumitem}
\usepackage{tabulary}
\usepackage{titlesec}
\usepackage{hyperref}

\hypersetup{colorlinks=true,linktoc=all,linkcolor=blue}

\newtheorem{lemma}{Lemma}
\newtheorem{theorem}{Theorem}
\newtheorem{proposition}[lemma]{Proposition}
\theoremstyle{definition}
\newtheorem{corollary}[lemma]{Corollary}
\newtheorem{example}[lemma]{Example}
\newtheorem{definition}[lemma]{Definition}

\newtheorem{remark}[lemma]{Remark}
\newtheorem{question}{Question}
\newtheorem{conjecture}{Conjecture}
\newtheorem{construction}{Construction}



\newcommand{\Z}{\mathbb Z}

\newcommand{\mbb}{\mathbb}
\newcommand{\mc}{\mathcal}


\DeclareMathOperator{\Spec}{Spec}
\DeclareMathOperator*{\holim}{holim}
\DeclareMathOperator*{\hocolim}{hocolim}
\DeclareMathOperator*{\colim}{colim}
\DeclareMathOperator*{\im}{im}
\DeclareMathOperator*{\coker}{coker}

\setcounter{secnumdepth}{4}
\titleformat{\paragraph}
{\normalfont\normalsize\bfseries}{\theparagraph}{1em}{}
\titlespacing*{\paragraph}
{0pt}{3.25ex plus 1ex minus .2ex}{1.5ex plus .2ex}





 \setlength{\headheight}{0pt}
  \lhead{Daniel, Brian, Tsutomu}\chead{UnivKThy Notes}\rhead{Summer 2017}
  \lfoot{}\cfoot{\thepage}\rfoot{}
  \pagestyle{fancy}


\begin{document}
I'm using the Arxiv version 5 Feb 2013

\tableofcontents

\section{Spectral Categories}

\subsection{Summary}
\subsubsection{Important definitions}
\begin{tabulary}{0.7 \textwidth}{|C|C|C|}
\hline
Term & Page Number & Loose Definition\\
\hline
\hline
Spectral category & 8& Category enriched over the category of
                       symmetric spectra.\\
\hline
$DK$ equivalence &9& Functor which induces stable equivalence of mapping spectra +
                       equivalence of ``homotopy'' categories.\\
\hline
Module over a spectral category &10& A spectral functor $A^{op}
                                     \rightarrow SymmSpt$ for a spectral
                                     category $A$.\\
\hline
Triangulated closure &10 & Yoneda embed $A \hookrightarrow \widehat
                           A^{cf}$ and take finite cell objects
                           (pushouts of coproducts).\\
\hline
Thick closure &11 & Same as above but take retracts of fintie cell objects.\\
\hline
Triangulated equivalence  & 11& Functor which induces $DK$ equivalence of
                                triangulated closures.\\
\hline
Morita equivalence of Spectral Categories & 11& Functor which induces $DK$ equivalence of thick closures.\\
\hline
\end{tabulary}

\subsubsection{Notation}
\begin{itemize}
\item $Cat_S$ is the category of small spectral categories and
  spectral functors.
\item $Cat_T$ is the category of small simplicial categories and
  simplicial functors.
\end{itemize}
\subsubsection{Key results}
\begin{enumerate}
\item There's a Quillen adjunction
\[
\xymatrix{Cat_S \ar@<.5ex>[r]^{\Sigma^\infty_+} & \ar@<.5ex>[l]^{\Omega^\infty} Cat_T}
\]

where $\Omega^\infty(F(A,B))$ is the zeroth space functor or
equivalently the simplicial
set $[n] \mapsto Hom_{Spt}(\mbb S \otimes \Delta^n,F(A,B)) \cong
Hom_{Spc}(\Delta^n,\Omega^\infty F(A,B)) \cong \Omega^\infty
F(A,B)_n$. This is the main theorem of Tabuada's paper ``Homotopy
Theory of Spectral Categories'' (see references). We can modify
$Cat_S$ up to weak equivalence to make this a simplicial Quillen
adjunction.

\begin{remark}
It might also be helpful to recall that, given a monoidal functor from
a monoidal category $M$ to a monoidal category $N$, any category
enriched over $M$ can be reinterpreted as a category enriched over
$N$. Furthermore, we recover the underlying category of an enriched
category by considering the functor $M(I,-) : M \rightarrow Set$ where
$I$ is the unit of the monoidal structure.
\end{remark}

\item There's a model category $\widehat{\mc A}$ of spectral $\mc
  A$-modules (these are defined as functors, so we put the projective
  model structure on the functor category). The referenced paper here
  is \href{/References/StabCatAreModuleCat.pdf}{Stable model categories are categories of modules} by Schwede
  and Shipley.
\end{enumerate}

\subsection{Clarifying Remarks}

\begin{itemize}
\item The definition of $A$-module is a generalization of the ordinary
  one. Recall that a ring is equivalently a preadditive category (an
  $Ab$-enriched category) with
  one object, and a module is a functor from this ring to abelian
  groups. Along the same vein, a ring spectrum is a spectral category with
  one object, and a module over this ring spectrum is a functor from
  this category to spectra.
\item A category of enriched functors is itself enriched. Recall that
  $Nat(F,G) = \int_C Hom(F-,G-)$. Via the theory of enriched ends we
  can define a mapping spectrum to be $\int_C A(F-,G-)$, where $A$
  denotes the mapping spectrum of two objects in our enriched
  category.
\item Given a functor $F : \mc A \rightarrow \mc B$, we get an obvious
  restriction functor $F^* : mod B := \widehat B \rightarrow \widehat A$
  given by precomposition with $F$ (recall the definition of a module
  as a functor!). It has a left adjoint functor $F_!$  given by an enriched
  coend. It sends an $A$-module $N$ to the coequalizer of
\[
\xymatrix{\bigvee_{o,p \in A} N(p) \wedge A(o,p) \wedge B(-,{F(o)})
  \ar@<.5ex>[r] \ar@<-.5ex>[r] & \bigvee_{o \in A} N(o) \wedge B(-,F(o))}.
\]

Recall that representable objects are ``free on one
generator''. Indeed, the usual definition of a free object is that
maps out of it are determined by maps out of a basis. We know, by the
Yoneda lemma, that maps out of representable objects are determined by
maps out of the identity morphism.

\item The paper omits the definition of a pretriangulated spectral
  category. It's in the
  \href{References/LocalizationTHH.pdf}{Mandell-Blumberg} paper in the
  references, definition 5.4.
\end{itemize}

\subsection{Background Material}

While the theory of infinity categories allows us to work
``coordinate-free'' in some sense, we first need some examples of
meaningful infinity categories and functors between them. One way to
get a slew of presentable $\infty$-categories is to take the underlying
infinity category of a left proper, simplicial, combinatorial model
category. We'll define the simplicial nerve after introducing
combinatorial model categories.

\subsubsection{Combinatorial Model Categories}
\begin{definition}
An infinite cardinal $\kappa$ is a \emph{regular cardinal} if it
satisfies the following property, which we think of as requiring the
collection of sets of smaller cardinality to be ``closed
under union'':
\begin{itemize}
\item no set of cardinality $\kappa$ is the union of fewer than
  $\kappa$ sets of cardinality less than $\kappa$.
\end{itemize}
\end{definition}

\begin{example}
$\aleph_0$, or the cardinality of $\mbb N$, is a regular
cardinal. This is because any set of cardinality less than $\aleph_0$
is finite, and no infinite set is a finite union of finite sets.
\end{example}

\begin{example}
Any successor cardinal is regular. For $\aleph_1$, this follows from
the fact that a countable union of countable sets is countable (we
need choice here).
\end{example}



\begin{definition}
Let $\kappa$ be an infinite regular cardinal. Then a
\emph{$\kappa$-filtered category} is one such that any diagram $F :D
\rightarrow C$, where $D$ has fewer than $\kappa$ morphisms admits an
extension $\widetilde F: D^+ \rightarrow C$ (i.e. $F$ has a
cocone). Here $D^+$ is the category obtained by freely adjoining a
terminal object to $D$.
\end{definition}

\begin{example}
Let $\kappa = \omega$ (or $\aleph_0$ in the notation we've been
using). Then we're requiring every FINITE diagram in $C$ to have a
cocone. This is equivalent to the usual definition of a filtered
category: a nonempty category s.t. each pair of objects has a join,
and for any parallel morphisms $f,g: c_1 \rightarrow c_2$ in $C$ there
exists a morphism $h : c_2 \rightarrow c_3$ such that $hf = hg$. In
other words, we can build a cocone for any finite diagram from these cocones.
\end{example}

\begin{example}
A preorder (there exists a unique morphism between any two objects) is
$\omega$-filtered precisely when it is \emph{directed}, i.e. any two objects
have a join.
\end{example}

\begin{definition}
Let $\kappa$ be a regular cardinal. Then an object $X$ such that
$C(X,-)$ commutes with $\kappa$-filtered colimits is called \emph{$\kappa$-compact}.
\end{definition}

\begin{definition}
An object $X$ of a category is \emph{small} if it is $\kappa$-compact
for some regular cardinal $\kappa$.
\end{definition}

\begin{definition}
A category $C$ is \emph{locally presentable} if
\begin{enumerate}
\item $C$ is a locally small category
\item $C$ has all small colimits
\item there exists a small set $S\hookrightarrow Obj(C)$ of
  $\lambda$-small objects that generates $C$ under $\lambda$-filtered
  colimits.
\item every obect in $C$ is a small object.
\end{enumerate}
\end{definition}

\begin{definition}
Let $C$ be a category and $I \subset Mor(C)$. Let \emph{$cell(I)$} be
  the class of morphisms obtained by transfinite composition of
  pushouts of coproducts of elements in $I$.
\end{definition}

\begin{definition}
A model category $C$ is \emph{cofibrantly generated} if there are
small sets of morphisms $I,J \subset Mor(C)$ such that
\begin{itemize}
\item $cof(I)$, the set of retracts of elements in $cell(I)$, is
  precisely the collection of cofibrations of $C$.
\item $cof(J)$ is precisely the collection of acyclic cofibrations in
  $C$; and
\item $I$ and $J$ permit the small object argument.
\end{itemize}
\end{definition}

\begin{definition}(Smith)
A model category $C$ is \emph{combinatorial} if it is

\begin{itemize}
\item locally presentable as a category, and
\item cofibrantly generated as a model category.
\end{itemize}


\end{definition}

\begin{example}
The category $sSet$ with the standand model structure on simplicial
sets is a combinatorial model category.
\end{example}

\begin{example}
The category $sSet$ with the Joyal model structure (so that the
quasi-categories are the fibrant objects) is combinatorial.
\end{example}

\begin{question}
Why should we care that a model category is combinatorial?
\end{question}




We'll define a left-proper model category, then give a fundamental
result of Dugger's:

\begin{definition}
A model category is \emph{left proper} if weak equivalence is
preserved by pushout along cofibrations.
\end{definition}

\begin{example}
A model category in which all objects are cofibrant is left
proper. This includes the standard model structure on simplicial sets,
as well the injective model structure on simplicial presheaves. This
follows from the Reedy lemma, which allows us to calculate homotopy
pushouts by considering diagrams s.t. the objects are cofibrant and
one of the maps is a cofibration (the point is that replacing this
diagram with a cofibrant diagram in the projective model structure on
diagrams is an acyclic cofibration of diagrams, not just a weak equivalence).
\end{example}

\begin{theorem}(Dugger)

Every combinatorial model category is Quillen equivalent to a left
proper simplicial combinatorial model category.
\end{theorem}


Now, we have the following extremely useful result on Bousfield
localizations:


\begin{theorem}
If $C$ is a left proper, simplicial, combinatorial model category, and
$S \subset Mor(C)$ is a small set of morphisms, then the left
Bousfield localization $L_SC$ does exist as a combinatorial model
category. Moreover, the fibrant objects of $L_SC$ are precisely the
$S$-local objects, and $L_S$ is left proper and simplicial.
\end{theorem}

In the context of infinity categories, we have some crucial
results of Lurie. For these to make sense, however, we need to
introduce the simplicial nerve construction. This is the
generalization of the ordinary nerve construction to simplicially
enriched categories.

\begin{definition}
We'll define a cosimplicial simplcially enriched category $S$. The
objects of $S[n]$ are $\{0,1,\dots,n\}$; the hom objects $S[n]_{i,j}
\in sSet$ for $i,j \in \{0,1,\dots,n\}$ are the nerves
\[
S[n](i,j) = N(P_{i,j})
\]
of the poset $P_{i,j}$ which is the poset of subsets of $[i,j]$ that
contain both $i$ and $j$ with partial order given by inclusion.
\end{definition}

\begin{definition}
The simplicial nerve of a simplicial category is the simplicial set
characterized by
\[
Hom_{sSet}(\Delta[n],N(C)) = Hom_{SSetCat}(S[n],C).
\]
\end{definition}

\begin{theorem}(HTT A.3.7.6)
Let $C$ be an $\infty$-category. The following conditions are
equivalent:

\begin{enumerate}
\item The $\infty$-category $C$ is presentable.
\item There exists a combinatorial simplicial model category $A$ and
  an equivalence $C \cong N(A^\circ)$.
\end{enumerate}

Here $A^{\circ}$ is the underlying category of bifibrant objects.
\end{theorem}

\begin{remark}(HTT A.3.7.7)

Let $A$ and $B$ be combinatorial simplicial model categories. Then the
underlying $\infty$-categories $N(A^\circ)$ and $N(B^\circ)$ are
equivalent iff $A$ and $B$ can be joined by a chain of simplicial
Quillen equivalences.
\end{remark}

\begin{theorem}(HTT 5.2.4.6)
Let $A$ and $B$ be simplicial model categories, and let
\[
\xymatrix{A \ar@<.5ex>[r]^F & \ar@<.5ex>[l]^G B}
\]

be a simplicial Quillen adjunction. This descends to an adjunction on
the underlying $\infty$-categories.
\end{theorem}








\section{Stable $\infty$-categories}

\subsection{Summary}
\subsubsection{Important definitions}
\begin{tabulary}{0.7 \textwidth}{|C|C|C|}
\hline
Term & Page Number & Loose Definition\\
\hline
\hline
Stable $\infty$ category & 14 & $\infty$-cat with finite (co)lims
                                s.t. hopushouts coincide with hopullbacks.\\
\hline
Idempotent complete & 14 & Image under Yoneda
                           embedding $C \rightarrow Fun(C^{op},Gpd_\infty)$ is closed under retracts.\\
\hline
Morita equivalence of stab $\infty$-cats& 14& Small stable
                                              $\infty$-cats $A,B$ are
                                              ME if Idem($A$) equiv to
  Idem$(B)$.\\
\hline
Spectrum Object in Pted $\infty$-cat & 15& A functor $F : N(\Z \times
                                           \Z) \rightarrow \mc C$
                                           s.t. certain diagrams are cartesian.\\
\hline
Spectral Yoneda embedding for stab $\infty$-cats & 15 & $C \approx
                                                        Sp(C_*)
                                                        \rightarrow
                                                        Sp(Fun(C^{op},Gpd_{\infty})_*)
  \approx Fun(C^{op},Sp(Gpd_{\infty}))$\\
\hline
Stably representable functor & 16 & Functor $C \rightarrow Spt$ equivalent to $Map(-,A)$
                                    for a spectrum object $A$, where
                                    $Map$ denotes mapping spectrum.\\
\hline
Accessible $\infty$-category & 17 & $\infty$-cat equivalent to $Ind_\kappa$ of
                                    a small $\infty$-cat.\\
\hline
Presentable $\infty$-category & 17 & $\infty$ cat generated under
                                     sufficiently large filtered
                                     colimits by some small $\infty$-cat.\\
\hline
Localization of $\infty$-categories & 18 & $\infty$-cat $C[S^{-1}]$
                                           equipped with a map $C
                                           \rightarrow C[S^{-1}]$
                                           which is universal for
                                           inverting elements of $S$.\\
\hline
Bousfield localization of presentable $\infty$-cat & 19 & Colimit
                                                          preserving
                                                          functor
                                                          whose right
                                                          adjoint is
                                                          fully faithful.\\
\hline
\end{tabulary}

\subsubsection{Notation}

\begin{itemize}
\item $Cat_\infty$ is the $\infty$-category of small $\infty$ categories.
\item $Cat^{ex}_\infty$ is the $\infty$-category of small stable $\infty$-categories and exact functors.
\item $Cat^{perf}_\infty$ is the $\infty$-category of small idempotent complete stable $\infty$-categories (and exact functors).
\end{itemize}

\subsubsection{Key results}

\begin{enumerate}
\item Given a pretriangulated spectral category $\mc C$, the
  $\infty$-category $N((Mod(\mc C))^{cf})$ is stable.

\item If $C$ is a pretriangulated spectral category, then $\pi_0 C$ admits a triangulated category structure.

\item If $C$ is a stable $\infty$-category, then $ho C$ admits a triangulated category structure.

\item The inclusion $Cat^{perf}_\infty \rightarrow Cat^{ex}_\infty$ has a left adjoint called idempotent completion.

\item If $C$ is a $\infty$-category with finite limits, there is a stable $\infty$-category $Stab(C)$ with a limit preserving functor $\Omega^\infty : Stab(C) \rightarrow C$. In particular, this is accomplished by taking $Stab(C) = Sp(C)$, the category of spectrum objects of $C$. If $C$ is already stable, then $\Omega^\infty$ is an equivalence, with inverse $\Sigma^\infty : C \rightarrow Sp(C)$.

\item If $C$ is a stable $\infty$-category, there is a \textit{spectral Yoneda embedding}
$$Y : C \simeq Sp(C_*) \rightarrow Sp(Fun(C^{op}, N(T)^{cf})_*) \simeq Fun(C^{op}, S_\infty)$$
with an adjoint \textit{mapping spectrum functor}
$$Map : C^{op} \times C \rightarrow S_\infty$$

\item If $C$ is a stable $\infty$-category, a $C^{op} \rightarrow S_\infty$ is stably representable if and only if it is represented by the suspension spectrum $\Sigma^\infty z$ of some $z \in C$, and this object is unique upto equivalence.

\end{enumerate}

\subsection{Clarifying Remarks}

\begin{enumerate}
\item Idempotent completion is unique upto equivalence by HTT Proposition 5.1.4.9. Existence is obtained by taking the full subcategory of $Pre(C)$ spanned by objects that are retracts of objects in the image of $C$ under the Yoneda embedding.
\item For this paper we assume stable $\infty$-categories to be pointed, i.e. they have zero objects. In the description of spectral Yoneda embedding above, $*$ indicates the subcategory of pointed objects.
\end{enumerate}

\subsection{Background Material}

\subsubsection{Pretriangulated Spectral Categories}

The material here is from Section 5 of \href{References/LocalizationTHH.pdf}{Mandell-Blumberg} (in the higher K-theory paper, they say Section 4 but that's either a typo or just not updated).

\begin{definition}
A spectral category $C$ is \textit{pretriangulated} if
\begin{enumerate}
\item There exists an object 0 of $C$ such that $C(-,0)$ is homotopically trivial. This means that it is weakly equivalent to the constant functor $*$ at the one point symmetric spectrum.
\item Whenever a $C$-module $M$ has the property that $\Sigma M$ is weakly equivalent to a representable $C$-module $C(-,c)$, then $M$ is weakly equivalent to some representable $C$-module $C(-,d)$.
\item Whenever the $C$-modules $M$ and $N$ are weakly equivalent to representables $C(-,a)$ and $C(-,b)$ respectively, the homotopy cofiber of any map of $C$-modules $M \rightarrow N$ is weakly equivalent to a representable $C$-module.
\end{enumerate}
\end{definition}

\begin{remark}
The first condition says that the homotopy category $\pi_0 C$ has a zero object. The second condition gives a desuspension functor on $\pi_0 C$ and the third condition gives a suspension functor on $\pi_0 C$.
\end{remark}

\begin{definition}
A spectral functor between spectral categories $F : C \rightarrow D$ is a \textit{DK-embedding} if for all objects $a,b$ in $C$, the induced map of spectra
$C(a,b) \rightarrow D(Fa,Fb)$ is a weak equivalence.
\end{definition}

\begin{remark}
In Blumberg-Mandell, a DK equivalence is a DK embedding satisfying one of the following equivalent conditions
\begin{enumerate}
\item For all object $d$ of $D$, there is an object $c$ of $C$ such that $D(-,d)$ and $D(-,Fc)$ are naturally isomorphic as $D$-modules.
\item The induced functor on the ``graded homotopy categorie'' $\pi_* C \rightarrow \pi_* D$ is an equivalence of categories.
\end{enumerate}
In Blumberg-Gepner-Tabuada, this second condition is relaxed to $\pi_0 C \rightarrow \pi_0 D$ being an equivalence of categories. This is because a spectral functor between pretriangulated spectral categories is a DK equivalence if and only if it is a DK embedding and the induced functor on $\pi_0$ is an equivalence of categories.
\end{remark}

\begin{theorem}(Blumberg-Mandell, Theorem 5.5)
Any small spectral category $C$ DK-embeds into a small pretriangulated spectral category.
\end{theorem}

\begin{remark}
We should think of this as taking the closure of $C$ under cofibration sequences and desuspensions in $C$-modules, via the Yoneda embedding. Note that this can be made functorial and it gives the ``minimal pretriangulated closure'' of $C$.
\end{remark}

\begin{definition}
A \textit{four-term Puppe sequence} in $\pi_0 C$ is a sequence of the form
$$a \rightarrow b \rightarrow c \rightarrow \Sigma a$$
if there exist a map of $C$-modules $f : M \rightarrow N$ such that the sequence
$$M \rightarrow N \rightarrow Cf \rightarrow \Sigma M$$
is isomorphic to the above sequence in the derived category of $C$-modules via the Yoneda embedding, and further the equvalence $\Sigma M \simeq C(-,\Sigma a) \simeq \Sigma C(-,a)$ is the suspension of the isomorphism $M \simeq C(-,a)$.
\end{definition}

\begin{theorem}(Blumberg-Mandell, Theorem 5.6)
Given a pretriangulated spectral category $C$, its homotopy category $\pi_0 C$ is triangulated with distinguished triangles the above four-term Puppe sequences.
\end{theorem}

\begin{proof}
Proof of Theorem 5.6 is just observing that $\pi_0 C$ embeds as a full subcategory of the homotopy category of $C$-modules (with projective model structure) and checking that it is closed under suspensions, desuspensions and distinguished triangles.
\end{proof}

\subsubsection{$\infty$-Categories}

\begin{definition} The \textbf{Joyal model structure} on simplicial sets is defined as follows :
\begin{itemize}
\item Cofibrations are levelwise monomorphisms.
\item Weak equivalences are \textit{(weak) categorical equivalence}. These are maps $f : A \rightarrow B$ such that for any $\infty$-category $X$, the map $X^B \rightarrow X^A$ induces an isomorphism on the fundamental category (the homotopy category) $ho C$.
\item Fibrations are determined by the above.
\end{itemize}
\end{definition}

\begin{remark}
All objects are cofibrant, and the fibrant objects are precisely $\infty$-categories.
\end{remark}

\begin{theorem}(HA Theorem 1.1.2.15) Let $C$ be a stable $\infty$-category. Then $ho C$ has a triangulated category structure with distinguished triangles coming from cofiber sequences.
\end{theorem}

\begin{remark}
There is a correspondance between pretriangulated spectral categories and stable $\infty$-categories. For example, if $C$ is a pretriangulated spectral category, then $N((ModC)^{cf})$ is a stable $\infty$-category.
\end{remark}

\subsubsection{Idempotent completion}

HTT 4.4.5 gives a good overview on the definition of idempotent completion for $\infty$-categories, while comparing it with the classical notion.

\begin{remark} The notion of retracts between classical and $\infty$-categorical settings are a bit different. An ordinary category $X$ is said to be \textit{idempotent complete} if every idempotent map $X \rightarrow X$ comes from some retract $Y$ of $X$. In such a situation $Y$ can be determined uniquely as an equalizer (or a coequalizer). Hence, if $C$ has finite limits or finite colimits, then $C$ is idempotent complete.

This is not the case for $\infty$-categories. Consider the category $C_*(R)$ consisting of bounded chain complex of finite rank free $R$-modules and consider $N(C_*(R))$, which is actuallt a stable $\infty$-category. Hence it admits finite limits and colimits, but it is idempotent complete if and only if every finitely generated projective $R$-module is stably free.

The problem is that an idempotent in an $\infty$-category shouldn't be just a morphism $e$ with $e \circ e \simeq e$ in $ho C$. It should specify homotopies on how to relate multiple compositions $e \circ e \circ ... \circ e \simeq e$. To achieve this, in HTT 4.4.5, simplicial sets called $Idem^+, Idem$ and $Ret$ are introduced. Now idempotents, weak retractions, strong retractions in $C$ are respectively defined to be maps of simplicial sets from $Idem, Ret, Idem^+$ to $C$. $C$ is \textit{idempotent complete} if every idempotent $F : Idem \rightarrow C$ has a colimit. This can be shown to be equivalent to the definition in the paper using results in HTT 5.1.4 and 5.1.5.
\end{remark}




















\section{Symmetric Monoidal Structures and Dualizable Objects}

\subsection{Summary}

\subsubsection{Important definitions}
\begin{tabulary}{0.7\textwidth}{|C|C|C|}
\hline
Term & pg & Loose Definition\\
\hline
\hline
Tensor Product for $\mathrm{Cat}^\mathrm{Perf}_\infty$ & 20 & If $\mathcal{A}$ and $\mathcal{B}$ are small stable idempotent-complete $\infty$-categories, we define $\mathcal{A} \widehat{\otimes} \mathcal{B} = (\mathrm{Ind}(\mathcal{A}) \otimes \mathrm{Ind}(\mathcal{B}))^\omega$. \\ \hline
Right-Compact Object & 21 & An object of $\mathrm{Fun}^\mathrm{ex}(\mathcal{A} \widehat{\otimes} \mathcal{B}^\mathrm{op} , \mathcal{S}_\infty)$ is right-compact if putting in some object $a \in \mathcal{A}$ in the left argument always yields a compact object of $\mathrm{Fun}^\mathrm{ex}(\mathcal{B}^\mathrm{op} , \mathcal{S}_\infty)$. \\ \hline
Proper Stable $\infty$-Category & 22 & Mapping spectra are compact. \\ \hline
Smooth Stable $\infty$-Category & 22 & Perfect as a bimodule over itself. If $\mathcal{A}$ is idempotent-complete, then it is smooth if and only if it is a representable $\mathcal{A}^{\mathrm{op}} \widehat{\otimes} \mathcal{A}$-module. (``Coherent''?) \\ \hline
Dualizable & 22 & An object of a symmetric monoidal $\infty$-category is dualizable if it is dualizable in the homotopy category. \\ \hline
\end{tabulary}


\subsubsection{Notation}

\begin{itemize}
    \item The $\infty$-category $\mathrm{Cat}_\infty^\mathrm{Perf}$ admits a symmetric monoidal structure. We write the tensor product as $\widehat{\otimes}$ to distinguish it from the usual tensor product of presentable stable $\infty$-categories.
\end{itemize}


\subsubsection{Key results}

\begin{enumerate}
    \item The $\infty$-category $\mathrm{Cat}_\infty^\mathrm{Perf}$ admits the structure of a closed symmetric monoidal $\infty$-category with tensor product given by $\widehat{\otimes}$. The unit is the $\infty$-category $\mathcal{S}_\infty^\omega$ of compact spectra. Internal $\mathrm{Hom}$ is given by $\mathrm{Fun}^\mathrm{ex}(\mathcal{A},\mathcal{B})$.

    \item For any small stable $\infty$-category $\mathcal{A}$, the stable Yoneda embedding $\mathcal{A} \to \mathrm{Fun}^\mathrm{ex}(\mathcal{A}^\mathrm{op}, \mathcal{S}_\infty)$ induces an equivalence $\mathrm{Ind}(A) \simeq \mathrm{Fun}^\mathrm{ex}(\mathcal{A}^\mathrm{op}, \mathcal{S}_\infty)$. In particular, one can model the idempotent-completion of $\mathcal{A}$ by the Yoneda embedding $\mathcal{A} \to \mathrm{Fun}^\mathrm{ex}(\mathcal{A}^\mathrm{op}, \mathcal{S}_\infty)^\omega$.

    \item Let $\mathcal{A}$ and $\mathcal{B}$ be small stable idempotent-complete $\infty$-categories. The functor category $\mathrm{Fun}^\mathrm{ex}(\mathcal{A},\mathcal{B})$ can be identified as a full subcategory of the $\infty$-category $\mathrm{Fun}^\mathrm{ex}(\mathcal{A}\widehat{\otimes} \mathcal{B}^\mathrm{op}, \mathcal{S}_\infty)$ of $\mathcal{A}^\mathrm{op} \widehat{\otimes} \mathcal{B}$-modules. In fact, it is the full subcategory spanned by right-compact $\mathcal{A}^\mathrm{op} \widehat{\otimes} \mathcal{B}$-modules. (Note, I believe there is a small typo in the second-to-last paragraph of page 21, where it says ``\ldots certain subcategory of $\mathrm{Fun}^\mathrm{L}(\mathcal{A}\widehat{\otimes} \mathcal{B}^\mathrm{op}, \mathcal{S}_\infty)$''; superscript should be $\mathrm{ex}$, not $\mathrm{L}$.)

    \item An object $\mathcal{A}$ of $\mathrm{Cat}_\infty^\mathrm{Perf}$ is dualizable (with respect to the symmetric monoidal structure on $\mathrm{Cat}_\infty^\mathrm{Perf}$) if and only if $\mathcal{A}$ is smooth and proper. Moreover, if $\mathcal{A}$ is dualizable, its dual is given by $\mathcal{A}^\mathrm{op}$.

\end{enumerate}


\subsection{Clarifying Remarks}

\begin{enumerate}
    \item This chapter assumes the monoidal structure on the
      $\infty$-category of presentable, stable $\infty$ categories,
      then defines some other monoidal products. The definition of the
      monoidal structure $\otimes$ on $\mc Pr_{St}^L$ is involved. In
      fact, it must be involved; the unit of this monoidal
      structure is the category $Sp$ of spectra. Hence, defining
      this monoidal structure gives us, in particular, a smash product
      of spectra.
      \item Key result (2) above is one step in showing that the
        definition of Morita equivalence via Idempotent completion
        matches the definition in terms of module categories. 
        \item There's a slight difference in notation: BGT denote a
          symmetric monoidal infinity category by $\mc C^{\infty}$,
          whereas below we define a symmetric monoidal infinity
          category asa coCartesian fibration $\mc C^{\infty}
          \rightarrow \mathscr F\mathrm{in}_*$.
          \item The terminology ``smooth'' and ``proper'' here comes
            from $dg$-categories (recall that this paper is a
            translation of Tabuada's work into the language of
            $\infty$-categories). In particular, a smooth proper $dg$
            category is one which closely resembles the category of
            perfect complexes on a smooth proper scheme, and these
            $dg$ categories can be characterized as the dualizable
            objects in some category of $dg$ categories.
\end{enumerate}


\subsection{Background Material}

\subsubsection{Symmetric Monoidal $\infty$-Categories}
The following can be found in chapter 2 of \href{References/LurieHigherAlgebra.pdf}{HA}. In particular, the introduction to this chapter is a very clear introduction to the main idea.

In ordinary category theory, a symmetric monoidal structure on a category $\mathcal{C}$ is usually described by a functor $\otimes : \mathcal{C} \times \mathcal{C} \to \mathcal{C}$, an identity object $1 \in \mathcal{C}$ and natural isomorphisms describing associativity, commutativity, and unitality. In this setting, one demands that these natural isomorphisms satisfy coherence conditions. If one tries to do something analogous in the setting of $\infty$-categories, one will quickly find that higher and higher coherence conditions must be imposed, to the point where this is prohibitively complicated. We will thus need to try something.

Another way to describe a symmetric monoidal structure on an ordinary category $\mathcal{C}$ is by a Grothendieck opfibration. Let $\mathscr{F}\mathrm{in}_*$ denote the category consisting of objects $\langle n \rangle = \{1, \ldots, n \} \sqcup \{*\}$ ($n \geq 0$) and functions between these sets that preserve $*$. By abuse, we will identify this category with the category of pointed finite sets. Some useful morphisms in $\mathscr{F}\mathrm{in}_*$ are the functions $\rho_i: \langle n \rangle \to \langle 1 \rangle$ given by
\[
    \rho_i(j) = \left\{
    \begin{tabular}{cl}
        $1$ & if $j=i$ \\
        $*$ & otherwise.
    \end{tabular}
    \right.
\]

Given a symmetric monoidal category $\mathcal{C}$, we can form a category $\mathcal{C}^\otimes$ in which
\begin{itemize}
    \item objects are finite (possibly empty) sequences of objects of $\mathcal{C}$, denoted by $[C_1, \ldots, C_n]$
    \item A morphism $f : [C_1, \ldots, C_n] \to [C_1', \ldots, C_m']$ consists of a subset $S \subseteq \{1, \ldots, n\}$, a map $\alpha : S \to \{1, \ldots, m\}$, and a collection of morphisms $\{f_j: \bigotimes_{\alpha(i)=j} C_i\to C_j'\}_{1 \leq j \leq m}$, and
    \item composition is defined in the only sensible way. See the introduction of chapter 2 of \href{References/LurieHigherAlgebra.pdf}{HA}.
\end{itemize}
We get a forgetful functor $\mathcal{C}^\otimes \to
\mathscr{F}\mathrm{in}_*$. In fact, this functor is a Grothendieck
opfibration. We can identify $\mathcal{C}$ with the fibre
$\mathcal{C}^\otimes_{\langle 1 \rangle}$ over $\langle 1 \rangle \in
\mathscr{F}\mathrm{in}_*$ . It has the special feature that the
functors $\mathcal{C}^\otimes_{\langle n \rangle} \to \mathcal{C}$
induced by $\rho_i : \langle n \rangle \to \langle 1 \rangle$ assemble
into an equivalence $\mathcal{C}^\otimes_{\langle n \rangle} \to
\mathcal{C}^n$. (This is often referred to as a ``Segal condition''.)
The main result is that a symmetric monoidal structure on
$\mathcal{C}$ can be recovered from such a Grothendieck
opfibration. Moreover, this construction generalizes readily to the
$\infty$-category setting, as demonstrated in the following
definition.

\begin{remark}
This Segal condition gives us functors $\mc C^2 \rightarrow \mc
C^{\otimes}_{\langle 2 \rangle} \xrightarrow{} \mc C$, which is how we
can recover the symmetric monoidal product from the category defined
above. So just to clarify, defining the category $\mc C^\otimes$ will
require us to specify what the objects $ C_1 \otimes \dots \otimes
C_j$ are. The point is that defining the bifunctor $\otimes$ by
specifying it in the form above lets us give a ``minimal
presentation'' of the coherence conditions.
\end{remark}

\begin{definition}
    A \emph{symmetric monoidal $\infty$-category} is a coCartesian fibration of simplicial sets $\mathcal{C}^\otimes \to \mathscr{F}\mathrm{in}_*$ such that, for each $n \geq 0$, the maps $\{\rho_i : \langle n \rangle \to \langle 1 \rangle \}_{1 \leq i \leq n}$ induce functors $\mathcal{C}^\otimes_{\langle n \rangle} \to \mathcal{C}^\otimes_{\langle 1 \rangle}$ that assemble into an equivalence $\mathcal{C}^\otimes_{\langle n \rangle} \to (\mathcal{C}^\otimes_{\langle 1 \rangle})^n$.
\end{definition}

Here, we make the usual abuse of notation of writing
$\mathscr{F}\mathrm{in}_*$ instead of
$N(\mathscr{F}\mathrm{in}_*)$. It should be noted that, given a
symmetric monoidal $\infty$-category $\mathcal{C}^\otimes \to
\mathscr{F}\mathrm{in}_*$, one typically thinks of
$\mathcal{C}^\otimes_{\langle 1 \rangle}$ as having been given a
symmetric monoidal structure.

\section{Morita Theory}

\subsection{Summary}

\subsubsection{Important definitions}
\begin{tabulary}{0.7\textwidth}{|C|C|C|}
\hline
Term & pg & Loose Definition\\
\hline
\hline
Stable simplicial category & 25 & Simplicial category s.t. the
                                  associated $\infty$-category is
                                  stable.\\
\hline
Stable spectral category & 25 & Spectral category whose associated
                                simplicial category is stable.\\
\hline
$\Psi_{tri}$ & 24 & Functor $N((Cat_S)^c)[W^{-1}] \rightarrow
                    N((Set_{\Delta})^c)[W^{-1}] \cong Cat_{\infty}$. 
                     Induced by the functor $Cat_S \rightarrow
                    Set_{\Delta}$ given by $A \mapsto \widehat A_{tri}
                    \mapsto N(\Omega^\infty(A)^{fib})$.\\
\hline
$\Psi_{perf}$ & 24 & Same as above, but replace $\widehat A_{tri}$
                     with $\widehat A_{perf}$.\\
\hline
Triangulated equivalence v2 & 30 & A map in the $\infty$-cat of small
                                   spectral categories
                                   s.t. $\Psi_{tri}f$ is an
                                   equivalence of stab $\infty$-cats.\\
\hline
Morita equivalence v2 & 30 & $\Psi_{perf}f$ is an equivalence.\\
\hline
rep$(B,A)$ & 31 & $\Upsilon(Fun^{ex}(B,A))$, the small pretriangulated
  spectral category associated to the small stable $\infty$-cat of
  exact functors from $B$ to $A$.\\
\hline
\end{tabulary}
\subsubsection{Notation}
\begin{itemize}
\item $\Upsilon$ is the right adjoint to $\Psi_{tri}$, and via
  inclusion to $\Psi_{perf}$.
\end{itemize}
\subsubsection{Key Results}
\begin{enumerate}
\item $\Psi_{tri}$ lands in $Cat^{ex}_{\infty} = $small stable $\infty$-cats.
\item $\Psi_{perf}$ lands in $Cat^{perf}_{\infty} = $idempotent
  complete small stable $\infty$-cats.
\item The $\infty$-category of stable $\infty$-cats is an accessible
  localization of the $\infty$-cat of spectral categories obtained by
  inverting the triangulated equivalences. In other words, the functor
  $\Psi_{tri}$ has a fully faithful and accessible right adjoint $\Upsilon$.
\item The $\infty$-category of stable idempotent complete $\infty$-cats is an accessible
  localization of the $\infty$-cat of spectral categories obtained by
  inverting the Morita equivalences.
\item The $\infty$-cats $Cat_\infty^{ex}$ and $Cat_\infty^{perf}$ are
  compactly generated, complete, and cocomplete.
\item Let $I$ be a small category. Given a diagram $\mc D$ of small
  stable $\infty$-categories indexed by $N(I)$, there exists and
  $I$-diagram of pretriangulated spectral categories $\widetilde{\mc D}$
  lifting $\mc D$.
\end{enumerate}
\subsection{Clarifying Remarks}
\begin{enumerate}
\item This whole section is basically setting up technical machinery to allow us
  to lift stable $\infty$-categories to spectral categories and make
  arguments with these more rigid objects. This is the content of (3)
  and (4) above. 
\item Given a small stable idempotent complete $\infty$-category $A$,
  we have that the conunit of the adjunction $\Psi_{perf}\Upsilon
  \rightarrow Id$ is a natural equivalence. Thus $A \cong
  \Psi_{perf}\Upsilon(A) \cong Idem\circ \Psi_{tri}\Upsilon(A)$, and
  recall that idempotent completion of a small stable $\infty$-cat can
  be modeled as $A \mapsto Fun^{ex}(A^{op},S_{\infty})^{\omega}$. It
  is in this sense that small stable idempotent complete
  $\infty$-categories are $\infty$-categories of modules. 
\end{enumerate}
\subsection{Background Material}
\subsubsection{Accessible Localizations}
\begin{lemma}
If a right adjoint is full and faithful, the counit is an isomorphism.
\end{lemma}

\begin{proof}
By definition of an adjunction $R \xrightarrow{\eta L} RLR
\xrightarrow{R\epsilon} R$ is the identity, so that $R\epsilon$ is an
isomorphism. Thus $\epsilon$ is an isomorphism since $R$ is fully
faithful. 
\end{proof}

\begin{remark}
The analogous result holds in the $\infty$-categorical setting.
\end{remark}

\begin{definition}
An $(\infty,1)$-functor $F : C \rightarrow D$ is accessible if $C$ is
an accessible $(\infty,1)$-category and there is a regular cardinal
$\kappa$ s.t. $F$ preserves $\kappa$-small filtered colimits.
\end{definition}

\begin{remark}
If an $(\infty,1)$-functor between accessible $(\infty,1)$-categories
has a left or right adjoint $(\infty,1)$-functor, then it is itself
accessible. 
\end{remark}

\begin{definition}
An $(\infty,1)$-functor $L: C \rightarrow C_0$ is called a
(reflective) localization of the $(\infty,1)$-category $C$ if it has a
right adjoint $(\infty,1)$-functor $i : C_0 \hookrightarrow C$ that is
full and faithful.
\end{definition}

\begin{definition}
A localization is accessible if the localization functor is an
accessible functor. 
\end{definition}



\section{Exact Sequences}

\subsection{Summary}

\subsubsection{Important definitions}
\begin{tabulary}{0.7\textwidth}{|C|C|C|}
\hline
Term & pg & Loose Definition\\
\hline
\hline
Verdier quotient $B/A$ & 32 & The cofiber of a fully faithful functor $f : A \rightarrow B$ in $Pr^L_{St}$\\
\hline
Exact sequence of presentable stable $\infty$-categories $A \rightarrow B \rightarrow C$ & 34 & The composite is trivial, $A \rightarrow B$ is fully faithful, $B/A \rightarrow C$ is an equivalence.\\
\hline
Exact sequence in $Cat_\infty^{ex(\kappa)}$ & 35 & If applying $Ind_\kappa(\_)$ gives an exact sequence in the above sense. \\
\hline
Split exact sequence $A \rightarrow B \rightarrow C$ in $Cat_\infty^{ex(\kappa)}$ & 37 & It is exact (in the above sense) and there exist a right adjoint $B \rightarrow A$  with unit of adjunction being a natural iso and a right adjoint $C \rightarrow B$ with counit of adjunction being a natural iso\\
\hline
Exact sequence of spectral categories & 38 & If applying $N(\Omega^\infty Mod(\_)^{cf})$ gives an exact sequence of presentable stable $\infty$-categories.\\
\hline
$Split(Cat_\infty^{ex})$ and $Split(Cat_\infty^{perf})$ & 39 & Subcategory of $Fun(\Delta^2, Cat_\infty^{ex})$ consisting of split exact sequences. Similarly for the other case. \\
\hline
Strict exact sequence of small stable $\infty$-categories & 40 & An exact sequence of the form $A \rightarrow B \rightarrow B/A$ such that $A \rightarrow B$ is the inclusion of a full subcategory and every object of $B$ that is a summand of an object of $A$ is also in $A$. \\
\hline
\end{tabulary}

\subsubsection{Notation}
\begin{itemize}
\item $Cat_\infty^{ex(\kappa)}$ is the $\infty$-category of $\kappa$-cocomplete stable $\infty$-categories and $\kappa$-small colimit preserving functors.
\end{itemize}

\subsubsection{Key Results}

\begin{enumerate}
\item If $A \rightarrow B$ is a fully faithful functor between presentable stable $\infty$-categories, the Verdier quotient $B/A$ is the Bousfield localization $B$ at the class of morphisms whose cones lie in the essential image of $A$. 
\item Let $A \rightarrow B$ be a fully faithful functor between presentable stable $\infty$-categories. Then $Ho(A)/Ho(B) \rightarrow Ho(A/B)$ is an equivalence. 
\item A functor between stable $\infty$-categories is fully faithful (\textit{resp.} an equivalence) if and only if the induced functor on the homotopy categories is fully faithful (\textit{resp.} an equivalence). No need to assume presentableness here. 
\item A sequence in $Cat_\infty^{ex(\kappa)}$ is exact if and only if the induced sequence on homotopy categories is exact in the classical sense. 
\item This proposition will be used later for the construction of nonconnective $K$-theory : let $A \rightarrow B \rightarrow C$ be an exact sequence of small stable $\infty$-categories. Then for any infinite regular cardinal $\kappa$, $Ind(A)^\kappa \rightarrow Ind(B)^\kappa \rightarrow Ind(C)^\kappa$ is an exact sequence of idempotent complete small stable $\infty$-categories.
\item $Split(Cat_\infty^{perf})$ is accessible. 
\item Any split exact sequence is equivalent to a strict exact sequence.
\end{enumerate}

\subsection{Clarifying Remarks}

\begin{enumerate}
\item In this section, a $\kappa$-continuous functor is defined to be one that preserves $\kappa$-filtered colimits. 
\item A fully faithful functor of stable $\infty$-categories descend to a fully faithful triangulated functor on their homotopy categories. 
\end{enumerate}

\subsection{Background Material}

\subsubsection{Classical notion of exactness}

\begin{remark}
All categories and functors we consider here are triangulated. 
\end{remark}

\begin{definition}
A sequence of functors $A \rightarrow B \rightarrow C$ is exact if the composite is zero, $A \rightarrow B$ is fully faithful, and the induced functor on the Verdier quotient $B/A \rightarrow C$ is cofinal. 
\end{definition}

\begin{definition}
If $A$ is a triangulated subcategory of $B$, then the Verdier quotient $B/A$ is the universal triangulated category with a functor $B \rightarrow B/A$ such that every object of $A$ is isomorphic to 0 under the functor. 
\end{definition}

\begin{definition}
A functor $C' \rightarrow C$ is cofinal if it becomes an equivalence after idempotent completion. Equivalently, if every object of $C$ is a summand of an object in the image. 
\end{definition}

\section{Additivity}

\subsection{Summary}

\subsubsection{Important definitions}
\begin{tabulary}{0.9\textwidth}{|C|C|C|}
\hline
Term & pg & Loose Definition\\
\hline
\hline
Additive invariant $E : Cat_\infty^{ex} \rightarrow D$ & 41 & A functor where $D$ is a stable presentable $\infty$-category, such that it inverts Morita equivalence (in the sense of Section 2), preserves filtered colimits, and is additive : every split exact sequence $A \rightarrow B \rightarrow C$ induces an equivalence $E(A) \vee E(C) \simeq E(B)$\\
\hline
\end{tabulary}

\subsubsection{Notation}
\begin{itemize}
\item $Fun_{add}(Cat_\infty^{ex},D)$ is the $\infty$-category of additive invariants with values in $D$. 
\item $Pre((Cat_\infty^{perf})^\omega)_*$ is the $\infty$-category $Fun(((Cat_\infty^{perf})^{\omega})^{op}, \mathcal{T}_{\infty,*})$, presheaves of pointed spaces. 
\item $\phi : Cat_\infty^{perf} \rightarrow Pre((Cat_\infty^{perf})^\omega)_*$ is defined by first applying the Yoneda embedding and then restricting the domain to $(Cat_\infty^{perf})^\omega$
\item $\mathcal{M}^{un}_{add}$ is the localization of $Pre((Cat_\infty^{perf})^\omega)_*$ with respect to the maps $\phi(B)/\phi(A) \rightarrow \phi(C)$ where $A \rightarrow B \rightarrow C$ is an element of a fixed set of representatives of split exact sequences in $(Cat_\infty^{perf})^\omega$. Let $\gamma$ denote the localization functor. 
\item $\mathcal{U}^{un}_{add}$ is the composite :
$$Cat_\infty^{ex} \overset{Idem(\_)}{\rightarrow} Cat_\infty^{perf} \overset{\phi}{\rightarrow} Pre((Cat_\infty^{perf})^\omega)_* \overset{\gamma}{\rightarrow} \mathcal{M}^{un}_{add}$$
\item $\mathcal{M}_{add}$ is the stabilization of $\mathcal{M}^{un}_{add}$. Recall this is obtained by taking the category of spectrum objects in $\mathcal{M}^{un}_{add}$. 
\item $\mathcal{U}_{add}$ is the composite
$$Cat_\infty^{ex} \overset{\mathcal{U}^{un}_{add}}{\rightarrow} \mathcal{M}^{un}_{add} \overset{\Sigma^{\infty}}{\rightarrow} \mathcal{M}_{add}$$
\end{itemize}

\subsubsection{Key Results}

\begin{enumerate}
\item ``Unstable universal additive invariant" : The functor $\mathcal{U}^{un}_{add}$ inverts Morita equivalences, preserves filtered colimits and sends split exact sequences to cofiber sequences. Moreover, $\mathcal{U}^{un}_{add}$ is universal with respect to these properties. For any presentable pointed $\infty$-category $D$, there is an equivalence of $\infty$-categories
$$Fun^L(\mathcal{M}^{un}_{add}, D) \simeq Fun^{un}_{add}(Cat_\infty^{ex},D)$$
where the RHS denotes the full subcategory of $Fun(Cat_\infty^{ex},D)$ of functors with the above listed properties. 
\item ``Stable universal additive invariant" : $\mathcal{U}_{add}$ is the unviersal additive invariant. Given any presentable stable $\infty$-category $D$, there is an equivalence of $\infty$-categories
$$Fun^L(\mathcal{M}_{add},D) \simeq Fun_{add}(Cat_\infty^{ex},D)$$
\end{enumerate}

\subsection{Clarifying Remarks}

\begin{enumerate}
\item Examples of additive invariants : analogues of algebraic $K$-theory and topological Hochschild cohomology. They will be constructed in later sections. 
\item There is an alternate description of the $\mathcal{M}_{add}$ by stablizing spaces first :
\begin{eqnarray*}
Stab(Pre((Cat_\infty^{perf})^\omega)_*) & = & Stab(Fun(((Cat_\infty^{perf})^{\omega})^{op}, \mathcal{T}_{\infty*,}))\\
& \simeq & Fun(((Cat_\infty^{perf})^{\omega})^{op}, Stab(\mathcal{T}_{\infty,*}))\\
& \simeq & Fun(((Cat_\infty^{perf})^{\omega})^{op}, S_\infty)\\
\end{eqnarray*}
So there is a natural functor $\psi : Cat_\infty^{perf} \rightarrow Stab(Pre((Cat_\infty^{perf})^\omega)_*)$. Then $\mathcal{M}_{add}$ can be described as the localization of $Stab(Pre((Cat_\infty^{perf})^\omega)_*)$ at $\psi(A)/\psi(B) \rightarrow \psi(C)$ for all split exact sequence representatives we considered earlier. 
\end{enumerate}

\subsection{Background Material}

It may be helpful to look at Chapter 1 Section 4 of \href{References/LurieHigherAlgebra.pdf}{HA} for details about stabilization and spectrum objects. 

\section{Connective $K$-Theory}

\subsection{Summary}

\subsubsection{Important definitions}
\begin{tabulary}{\textwidth}{|C|C|C|}
\hline
Term & pg & Loose Definition\\
\hline
\hline
Gap$([n],\mc C)$ (Waldhausen construction)  & 44 & Full subcat of $Fun(N(Ar[n]),\mc C)$ spanned
                        by $F$ s.t. $F(i,i) = 0$
                        and if $i < j < k$, then $F(j,k) =
                        p.o. (F(j,j) \leftarrow F(i,j) \rightarrow F(i,k))$.\\
\hline


\end{tabulary}

\subsubsection{Notation}
\begin{itemize}
\item Ar$[n]$ is the  category of arrows in $[n]$.

\item $S^\infty_{\bullet} \mc C$  is the  simplicial $\infty$-cat defined by
                                  $S_n^\infty \mc C = Gap([n],\mc
                                  C)$.

\item $\Omega |(S^\infty_{\bullet} \mc C)_{iso}|$  is  the
                                                   $\infty$-categorical
                                                   version of
                                                   Waldhausen's
                                                   $K$-theory
                                                   space. Here iso
                                                   means core. 

\item $(S_{\bullet}^\infty)^n$  is  the $n$th iteration of the
                                $S_{\bullet}$ construction. Makes sense
                                because the output is a pted
                                $\infty$-cat with fintie colimts. 


\item $|((S_{\bullet}^\infty)^n(\mc C))_{iso}|$  denotes  the levels of the
                                                 Waldhausen $K$-theory
                                                 spectrum. 
\item Given a small stable $\infty$-cat $\mc A$, $K^\omega_{\mc A}$
  is the object
\[
\mc B \mapsto |(S_{\bullet}^\infty (Fun^{ex}(\mc B, Idem(\mc A))))_{iso}|
\]
in $Pre((Cat_{\infty}^{perf})^{\omega})_*$ and $K_{\mc A}$ is the
object
\[
\mc B \mapsto K(Fun^{ex}(\mc B,Idem(\mc A)))
\]
in $Pre_{S_{\infty}}((Cat_{\infty}^{perf})^{\omega})$. 
\end{itemize}
\subsubsection{Key Results}
\begin{enumerate}
\item Let $\mc C$ be an $\infty$-category with finite colimits. Then
  for each $n$, the forgetful functor
\[
Gap([n],\mc C) \rightarrow Fun(\Delta^{1,2,\dots,n},\mc C)
\]
is an equivalence of $\infty$-categories (and $\Delta^{1,2,\dots,b}
\approx N([n-1])$).
\item Let $\mc A$ be a combinatorial simplicial model category and $C
  \subset \mc A$ a full subcategory. Then for each $n$, the induced
  map
\[
N(C^{Ar[n]})^{cf}) \rightarrow Fun(N(Ar[n]),N(C^{cf}))
\]
is a categorical equivalence of simplicial sets.

\item Let $\mc A$ be a simplicial model category and $C \subset \mc A$
  a small full subcategory of the cofibrants which admits all homotopy
  pushouts and is a Waldhausen category via the model structure on
  $\mc A$. Then there is an equivalence of spectra
\[
K(C) \cong K(N((C)^{cf}))
\]
which is natural is weakly exact functors. 


\item A corollary of the above is: Let $\mc C$ be a small
  pretriangulated spectral category and let $\mc M_C$ denote the
  category of perfect $\mc C$-modules with its Waldhausen structure
  induced by the model structure on $\mc C$-modules. Then there is an
  isomorphism in the stable category
\[
K(\mc M_C) \cong K(\psi_{perf}\mc C).
\]

This is key because every category in $Cat_{\infty}^{perf}$ is
equivalent to $\psi_{perf}\mc C$ for some spectral category $\mc C$ by
section 4 results.
\item The algebraic $K$-theory functor
\[
K : Cat_{\infty}^{perf} \rightarrow S_{\infty}
\]
is an additive invariant.
\item Let $\mc A$ be a small stable $\infty$-category and $\mc B$ be a
  compact idempotent-complete small stable $\infty$-category. Then
  there is a natural equivalence of spectra
\[
Map(\mc U_{add}(\mc B),\mc U_{add}(\mc A)) \cong K(Fun^{ex}(\mc
B,Idem(\mc A))).
\]
When $\mc B$ is the small stable $\infty$-cat $S^\omega_{\infty}$ of
compact spectra, there is a natural equivalence of spectra
\[
Map(\mc U_{add}(S^\omega),\mc U_{add}(\mc A)) \cong K(Idem(\mc A)).
\]
In particular, we have isomorphisms of abelian groups
\[
Hom(\mc U_{add}(S^\omega_{\infty}),\Sigma^{-n}\mc U_{add}(\mc A))
\cong K_n(Idem(\mc A))
\]
in the triangulated category $Ho(\mc M_{add})$.
\item Let $\mc A$ be a small stable $\infty$-category. Then, we have a
  natural equivalence $\Sigma (\mc U^{un}_{add}(\mc A)) \cong K_{\mc
    A}^{\omega}$ in $\mc M_{add}^{un}$ and a natural equivalence $\Sigma
  \mc U_{add}(\mc A) \cong \Sigma K_{\mc A}$ in $\mc M_{add}$.
\item Let $\mc A$ be a small stable $\infty$-category. Then the
  presheaves $K_{\mc A}^{\omega}$ and $K_{\mc A}$ are local,
  i.e. given any split exact sequence $\mc B \rightarrow \mc C
  \rightarrow \mc D$ in $\mc E$, the induced maps of spectra
\[
map(\phi(\mc D),K_{\mc A}^\omega) \xrightarrow{\sim} Map(\phi(\mc
C)/\phi(\mc A), K_{\mc A}^\omega)
\]
\[
map(\psi(\mc D),K_{\mc A}) \xrightarrow{\sim} Map(\psi(\mc
C)/\psi(\mc A), K_{\mc A}^\omega)
\]
are equivalences. 
\end{enumerate}
\subsection{Clarifying Remarks}
\begin{itemize}
\item The structure maps in the $K$-theory spectrum $S^1 \wedge
  (C_{iso}) \rightarrow |(S_{\bullet}^\infty \mc C)_{iso}|$ are given
  by the following facts: 
\begin{enumerate}
\item$Gap([1],\mc C)$ is equivalent to $\mc C$
\item $Gap([0], \mc C)$ is equivalent to $pt$
\item The realization of a simplicial space  $K_n$ whose 0-simplices are a
  point contains a copy of $S^1 \wedge K_1$ (just think about the
  formula, we product with an interval, then crush the two ends).
\end{enumerate}
\item The corepresentability of $K$-theory says that, as a functor
  $Cat_{\infty}^{perf} \rightarrow S_{\infty}$, $K$ is equivalent to
  the composite functor $Map(U_{add}(S_\infty^\omega), -) \circ U_{add}$. In
  particular, under the equivalence $(U_{add})^* : Fun^L(\mc
  M_{add},S_\infty) \xrightarrow{\sim} Fun_{add}(Cat_\infty^{perf},S_\infty)$,
  we can think of $K$ as the colimit preserving functor
  $Map(U_{add}(S_\infty^\omega),-) : \mc M_{add} \rightarrow
  S_\infty$. Note that I'm restring to idempotent complete categories
  because I don't want to add the idempotent completion functor to the
  composite above. 
\item To summarize the results of this section: $K$ theory is an
  additive invariant valued in spectra. Thus, it corresponds to a
  colimit preserving functor $\mc M_{add} \rightarrow S_{\infty}$. The
  functor corresponding to $K$ is the representable functor $Map(U_{add}(S^\omega_\infty),-)$.
\end{itemize}
\subsection{Background Material}

\subsubsection{Arrow Categories}

See \href{References/quasicats.pdf}{Charles' Notes} section 7.6 for a
description of the twisted arrow category, and note the remark just
before 7.7: we have to consider spans such that all squares in the
span are pullback squares if we want a quasi-category.   

To get the arrow category from the twisted arrow category, we just
flip all the arrows pointing left. Once we've done this, it's pretty
clear pictorially what the nerve will be. We'll do this process in an
example below:

The category $[2]^{tw}$ can be visualized as:

\[
\xymatrix{ & & 02 \ar[dr] \ar[dl] & & \\ & 01 \ar[dl] \ar[dr]& &12
  \ar[dr]\ar[dl] & \\ 00& & 11& & 22}.
\]

Now we flip the arrows pointing left to obtain $Ar([2])$:


\[
\xymatrix{ & & 02 \ar[dr]  & & \\ & 01 \ar[ur] \ar[dr]& &12
  \ar[dr] & \\ 00 \ar[ur] & & 11 \ar[ur]& & 22}.
\]

Filling in the composites, we get

\[
\xymatrix{ & & 02 \ar[dr]  & & \\ & 01 \ar[rr]\ar[ur] \ar[dr]& &12
  \ar[dr] & \\ 00 \ar[rr] \ar[ur] & & 11 \ar[rr] \ar[ur]& & 22}.
\]

making it more or less clear what the non-degenerate simplices of
$N(Ar[2))$ are. 

Now a functor out of this category can be written as: 
\[
\xymatrix{ & & X_{02} \ar[dr]  & & \\ & X_{01} \ar[ur] \ar[dr]& & X_{12}
  \ar[dr] & \\ X_{00} \ar[ur] & & X_{11} \ar[ur]& & X_{22}}.
\]

When we define $Gap([2],\mc C)$, we have the additional requirements
that the middle square is cocartesian and that $X_{00},X_{11},X_{22}$
are the zero object. In particular, up to a contractible space of
choices, the only data we need to specify
is $X_{01} \rightarrow X_{02}$. Mapping such a functor to the functor
obtained by restricting to the subcategory $01 \rightarrow 02$ yields
a map $Gap([2],\mc C) \rightarrow Fun(\Delta^{1,2},\mc C)$ which is
intuitively an equivalence. 

\subsubsection{$K$-theory via group completion}

This is from
\href{References/CategoriesAndCohomologyTheories.pdf}{Segal}. Given a
$\Gamma$-space $A$ (see the background material of section 3 above), we
can associate to it a spectrum with component spaces $A(1), BA(1),
B^2A(1), \dots $, where $1$ is the set $\{1\}$, and $BA$ is the
$\Gamma$ space s.t. for any finite set $S$, $BA(S)$ is the realization
of the $\Gamma$-space $T \mapsto A(S \times T)$. The key here is that
$B(-)$ of a $\Gamma$-space is again a $\Gamma$ space, so that we can
iterate the construction. 

The important example to keep in mind is the $\Gamma$-space $A$ which
is $A(1) = \coprod BGL_n$, $A(2) = \coprod_{n \geq 0} (EG_m \times
EG_n \times EG_{m+n})/(G_m \times G_n) , \dots$.

One important remark in the paper is that an $H$-space $X$ has a
homotopy inverse if it's grouplike and it has a numerable covering by
sets which are contractible in $X$. The realization of a simplicial
space has such a numerable covering if the space of $0$-simplices is
contractible. 

Now $BA(1)$ is the realization of $A$, and by definition of a
$\Gamma$-space, $A_0$ is contractible. Thus the product on $B^kA(1)$
has a homotopy inverse for any $k \geq 1$, and hence (by prop 1.4 in
the paper), $B^kA(1) \rightarrow \Omega B^{k+1}A(1)$ is a homotopy
equivalence. Thus the spectrum we formed above is ALMOST an
$\Omega$-spectrum, but it isn't necessarily true that $A(1)\rightarrow
\Omega BA(1)$ is an equivalence. The spectrification of this
prespectrum will be given by $\Omega BA(1), BA(1), B^2A(1),\dots $, so
we'd like to know what the precise relationship between $\Omega BA(1)$
and $A(1)$ is. Indeed, this is asking for the comparison between the
monoid $[X,A(1)]$ and the zeroth cohomology group $[X,\Omega
BA(1)]$. Indeed, going back to our above example, we see that
$[X,A(1)]$ should be the monoid of vector bundles, so we'd hope that
$\Omega BA(1)$ would be the group completion of this monoid. The
upshot is that in nice cases (the ones we care about) this is true. 

In
particular, given a $\Gamma$-spaces $A$, one can naturally associate
another $\Gamma$-space $A'$ with a map $A \rightarrow A'$ with the
following properties:
\begin{itemize}
\item $\pi_0(A')$ is the abelian group associated to the monoid
  $\pi_0(A)$; and
\item $BA \rightarrow BA'$ is a weak equivalence of spectra. 
\end{itemize}

\subsubsection{Chunks}
\begin{definition}
Say that a model category $\mc S$ is \emph{excellent} if it is
equipped with a symmetric monoidal structue and satisfies the
following conditions:
\begin{enumerate}
\item The model category $\mc S$ is combinatorial.
\item Every monomorphism in $\mc S$ is a cofibration, and the
  collection of cofibrations is stable under products.
\item THe collection of weak equivalences in $\mc S$ is stable under
  filtered colimits.
\item The monoidal structure on $\mc S$ is compatible with the model
  structure. In other words, the tensor product functor $\otimes : \mc
  S \times \mc S \rightarrow \mc S$ is a left Quillen bifunctor.
\item The monoidal model category $\mc S$ satisfies the invertibility
  hypothesis. 
\end{enumerate}

\begin{remark}
The invertibility hypothesis essentially says that inverting a
morphism $f$ in an $\mc S$-enriched category $C$ does not change the
homotopy type of $C$ when $f$ is already invertible up to homotopy. 
\end{remark}
\begin{example}
The canonical example is the category of simplicial sets when endowed
with the Kan model structure and cartesian product. 
\end{example}
\end{definition}
\begin{definition}
Let $\mc S$ be an excellent model category, and let $A$ be combinatorial
$\mc S$-enriched model category. A \emph{chunk} of $\mc A$ is a full
subcategory  $ \mc U \subset \mc A$ with the following properties:
\begin{enumerate}
\item Let $A$ be an object of $\mc U$ and let $\{\phi_i : A
  \rightarrow B_i\}_{i \in I}$ be a finite collection of morphisms in
  $\mc U$. Then there exists a factorization
\[
A \xrightarrow{p} \overline A \xrightarrow{q} \prod_{i \in I} B_i
\]
of the product map $\prod_{i \in I} \phi_i$, where $p$ is a trivial
cofibration, $q$ is a fibration, and $\overline A \in \mc
U$. Moreover, this factorization can be chosen to depend functorially
on the collection $\{\phi_i\}$ via an $\mc S$-enriched functor.
\item Let $A$ be an object of $\mc U$ and let $\{\phi_i : B_i
  \rightarrow A\}_{i \in I}$ be a finite collection of morphisms in
  $\mc U$. Then there exists a factorization
\[
\coprod_{i \in I} B_i \xrightarrow{p} \overline A \xrightarrow{q} A
\]
of the coproduct map $\coprod_{i \in I} \phi_i$, where $p$ is a
cofibration, $q$ is a trivial fibration, and $\overline A \in \mc
U$. Moreover, this factorization can be chosen to depend functorially
on the collection $\{\phi_i\}$ via an $\mc S$-enriched functor. 
\end{enumerate}
\end{definition}

\begin{definition}
Let $\mc S$ be an excellent model category, $\mc A$ a combinatorial
$\mc S$-enriched model category, and $C$ an $\mc S$-enriched
category. We will say that a full subcategory $\mc U \subset \mc A$ is
a $C$-chunk of $\mc A$ if it is a chunk of $\mc A$ and the subcategory
$\mc U^C$ is a chunk of $\mc A^C$. Here we regards $\mc A^C$ as
endowed with the projective model structure. 
\end{definition}

The main theorem about chunks is the following ridigification theorem:

\begin{theorem}(HTT 4.2.4.4)
Let $S$ be a small simplicial set, $\mc C$ a small simplicial
category, and $u : \mathfrak C[S] \rightarrow \mc C$ an
equivalence. Suppose that $\mathbf{A}$ is a combinatorial simplicial
model category and let $\mc U$ be a $\mc C$-chunk of
$\mathbf{A}$. Then the induced map
\[
N((\mc U^{\mc C})^{\circ}) \rightarrow Fun(S,N(\mc U^{\circ}))
\]
is a categorical equivalence of simplicial sets. 
\end{theorem}

This is a rigidification in the sense that the category on the left is
the nerve of a category of honest (simplicial) functors.

\end{document}
