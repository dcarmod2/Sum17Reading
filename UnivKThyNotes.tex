\documentclass[letterpaper]{article}
\usepackage{fancyhdr}
\usepackage[margin=1in]{geometry}
\usepackage{amsmath,amsthm,amssymb,amsfonts,amsopn,mathrsfs,mathtools}
%\usepackage{parskip}
\usepackage{tikz}
\usetikzlibrary{arrows}
\usepackage[all]{xy}
\usepackage{pdfpages}
\usepackage[compact]{titlesec}
\usepackage[normalem]{ulem}
\usepackage{cite}
\usepackage{hyperref}
\usepackage{color}
\usepackage{enumitem}
\usepackage{tabulary}
\usepackage{titlesec}



\newtheorem{lemma}{Lemma}
\newtheorem{theorem}{Theorem}
\newtheorem{proposition}[lemma]{Proposition}
\theoremstyle{definition}
\newtheorem{corollary}[lemma]{Corollary}
\newtheorem{example}[lemma]{Example}
\newtheorem{definition}[lemma]{Definition}

\newtheorem{remark}[lemma]{Remark}
\newtheorem{question}{Question}
\newtheorem{conjecture}{Conjecture}
\newtheorem{construction}{Construction}



\newcommand{\Z}{\mathbb Z}

\newcommand{\mbb}{\mathbb}
\newcommand{\mc}{\mathcal}


\DeclareMathOperator{\Spec}{Spec}
\DeclareMathOperator*{\holim}{holim}
\DeclareMathOperator*{\hocolim}{hocolim}
\DeclareMathOperator*{\colim}{colim}
\DeclareMathOperator*{\im}{im}
\DeclareMathOperator*{\coker}{coker}

\setcounter{secnumdepth}{4}
\titleformat{\paragraph}
{\normalfont\normalsize\bfseries}{\theparagraph}{1em}{}
\titlespacing*{\paragraph}
{0pt}{3.25ex plus 1ex minus .2ex}{1.5ex plus .2ex}





 \setlength{\headheight}{0pt}
  \lhead{Daniel, Brian, Tsutomu}\chead{UnivKThy Notes}\rhead{Summer 2017}
  \lfoot{}\cfoot{\thepage}\rfoot{}
  \pagestyle{fancy} 


\begin{document}
I'm using the Arxiv version 5 Feb 2013

\section{Spectral Categories}

\subsection{Summary}
\subsubsection{Important definitions}
\begin{tabular}{|c|c|}
\hline
Definition & Page Number\\
\hline
\hline
Spectral category & 8\\
\hline
$DK$ equivalence &9\\
\hline
Module over a spectral category &10\\
\hline
Triangulated closure &10 \\
\hline
Triangulated equivalence  & 11\\
\hline
Morita equivalence & 11\\
\hline
\end{tabular}

\subsubsection{Notation}
\begin{itemize}
\item $Cat_S$ is the category of small spectral categories and
  spectral functors. 
\item $Cat_T$ is the category of small simplicial categories and
  simplicial functors. 
\end{itemize}
\subsubsection{Key results}
\begin{enumerate}
\item There's a Quillen adjunction
\[
\xymatrix{Cat_S \ar@<.5ex>[r]^{\Sigma^\infty_+} & \ar@<.5ex>[l]^{\Omega^\infty} Cat_T}
\]

where $\Omega^\infty(F(A,B))$ is the zeroth space functor or
equivalently the simplicial
set $[n] \mapsto Hom_{Spt}(\mbb S \otimes \Delta^n,F(A,B)) \cong
Hom_{Spc}(\Delta^n,\Omega^\infty F(A,B)) \cong \Omega^\infty
F(A,B)_n$. This is the main theorem of Tabuada's paper ``Homotopy
Theory of Spectral Categories'' (see references). We can modify
$Cat_S$ up to weak equivalence to make this a simplicial Quillen
adjunction.

\begin{remark}
It might also be helpful to recall that, given a monoidal functor from
a monoidal category $M$ to a monoidal category $N$, any category
enriched over $M$ can be reinterpreted as a category enriched over
$N$. Furthermore, we recover the underlying category of an enriched
category by considering the functor $M(I,-) : M \rightarrow Set$ where
$I$ is the unit of the monoidal structure.
\end{remark}

\item There's a model category $\widehat{\mc A}$ of spectral $\mc
  A$-modules (these are defined as functors, so we put the projective
  model structure on the functor category).
\end{enumerate}

\subsection{Background Material}

While the theory of infinity categories allows us to work
``coordinate-free'' in some sense, we first need some examples of
meaningful infinity categories and functors between them. One way to
get a slew of presentable $\infty$-categories is to take the underlying
infinity category of a left proper, simplicial, combinatorial model
category. We'll define the simplicial nerve after introducing
combinatorial model categories.

\subsubsection{Combinatorial Model Categories}
\begin{definition}
An infinite cardinal $\kappa$ is a \emph{regular cardinal} if it
satisfies the following property, which we think of as requiring the
collection of sets of smaller cardinality to be ``closed
under union'':
\begin{itemize}
\item no set of cardinality $\kappa$ is the union of fewer than
  $\kappa$ sets of cardinality less than $\kappa$.
\end{itemize}
\end{definition}

\begin{example}
$\aleph_0$, or the cardinality of $\mbb N$, is a regular
cardinal. This is because any set of cardinality less than $\aleph_0$
is finite, and no infinite set is a finite union of finite sets.
\end{example}

\begin{example}
Any successor cardinal is regular. For $\aleph_1$, this follows from
the fact that a countable union of countable sets is countable (we
need choice here).
\end{example}



\begin{definition}
Let $\kappa$ be an infinite regular cardinal. Then a
\emph{$\kappa$-filtered category} is one such that any diagram $F :D
\rightarrow C$, where $D$ has fewer than $\kappa$ morphisms admits an
extension $\widetilde F: D^+ \rightarrow C$ (i.e. $F$ has a
cocone). Here $D^+$ is the category obtained by freely adjoining a
terminal object to $D$.
\end{definition}

\begin{example}
Let $\kappa = \omega$ (or $\aleph_0$ in the notation we've been
using). Then we're requiring every FINITE diagram in $C$ to have a
cocone. This is equivalent to the usual definition of a filtered
category: a nonempty category s.t. each pair of objects has a join,
and for any parallel morphisms $f,g: c_1 \rightarrow c_2$ in $C$ there
exists a morphism $h : c_2 \rightarrow c_3$ such that $hf = hg$. In
other words, we can build a cocone for any finite diagram from these cocones.
\end{example}

\begin{example}
A preorder (there exists a unique morphism between any two objects) is
$\omega$-filtered precisely when it is \emph{directed}, i.e. any two objects
have a join.
\end{example}

\begin{definition}
Let $\kappa$ be a regular cardinal. Then an object $X$ such that
$C(X,-)$ commutes with $\kappa$-filtered colimits is called \emph{$\kappa$-compact}.
\end{definition}

\begin{definition}
An object $X$ of a category is \emph{small} if it is $\kappa$-compact
for some regular cardinal $\kappa$.
\end{definition}

\begin{definition}
A category $C$ is \emph{locally presentable} if 
\begin{enumerate}
\item $C$ is a locally small category
\item $C$ has all small colimits
\item there exists a small set $S\hookrightarrow Obj(C)$ of
  $\lambda$-small objects that generates $C$ under $\lambda$-filtered
  colimits.
\item every obect in $C$ is a small object.
\end{enumerate}
\end{definition}

\begin{definition}
Let $C$ be a category and $I \subset Mor(C)$. Let \emph{$cell(I)$} be
  the class of morphisms obtained by transfinite composition of
  pushouts of coproducts of elements in $I$.
\end{definition}

\begin{definition}
A model category $C$ is \emph{cofibrantly generated} if there are
small sets of morphisms $I,J \subset Mor(C)$ such that
\begin{itemize}
\item $cof(I)$, the set of retracts of elements in $cell(I)$, is
  precisely the collection of cofibrations of $C$.
\item $cof(J)$ is precisely the collection of acyclic cofibrations in
  $C$; and
\item $I$ and $J$ permit the small object argument. 
\end{itemize}
\end{definition}

\begin{definition}(Smith)
A model category $C$ is \emph{combinatorial} if it is

\begin{itemize}
\item locally presentable as a category, and 
\item cofibrantly generated as a model category.
\end{itemize}


\end{definition}

\begin{example}
The category $sSet$ with the standand model structure on simplicial
sets is a combinatorial model category. 
\end{example}

\begin{example}
The category $sSet$ with the Joyal model structure (so that the
quasi-categories are the fibrant objects) is combinatorial.
\end{example}

\begin{question}
Why should we care that a model category is combinatorial?
\end{question}




We'll define a left-proper model category, then give a fundamental
result of Dugger's:

\begin{definition}
A model category is \emph{left proper} if weak equivalence is
preserved by pushout along cofibrations.
\end{definition}

\begin{example}
A model category in which all objects are cofibrant is left
proper. This includes the standard model structure on simplicial sets,
as well the injective model structure on simplicial presheaves. This
follows from the Reedy lemma, which allows us to calculate homotopy
pushouts by considering diagrams s.t. the objects are cofibrant and
one of the maps is a cofibration (the point is that replacing this
diagram with a cofibrant diagram in the projective model structure on
diagrams is an acyclic cofibration of diagrams, not just a weak equivalence).
\end{example}

\begin{theorem}(Dugger)

Every combinatorial model category is Quillen equivalent to a left
proper simplicial combinatorial model category. 
\end{theorem}


Now, we have the following extremely useful result on Bousfield
localizations:


\begin{theorem}
If $C$ is a left proper, simplicial, combinatorial model category, and
$S \subset Mor(C)$ is a small set of morphisms, then the left
Bousfield localization $L_SC$ does exist as a combinatorial model
category. Moreover, the fibrant objects of $L_SC$ are precisely the
$S$-local objects, and $L_S$ is left proper and simplicial.
\end{theorem}

In the context of infinity categories, we have some crucial
results of Lurie. For these to make sense, however, we need to
introduce the simplicial nerve construction. This is the
generalization of the ordinary nerve construction to simplicially
enriched categories.

\begin{definition}
We'll define a cosimplicial simplcially enriched category $S$. The
objects of $S[n]$ are $\{0,1,\dots,n\}$; the hom objects $S[n]_{i,j}
\in sSet$ for $i,j \in \{0,1,\dots,n\}$ are the nerves
\[
S[n](i,j) = N(P_{i,j})
\]
of the poset $P_{i,j}$ which is the poset of subsets of $[i,j]$ that
contain both $i$ and $j$ with partial order given by inclusion.
\end{definition}

\begin{definition}
The simplicial nerve of a simplicial category is the simplicial set
characterized by 
\[
Hom_{sSet}(\Delta[n],N(C)) = Hom_{SSetCat}(S[n],C).
\]
\end{definition}

\begin{theorem}(HTT A.3.7.6)
Let $C$ be an $\infty$-category. The following conditions are
equivalent:

\begin{enumerate}
\item The $\infty$-category $C$ is presentable.
\item There exists a combinatorial simplicial model category $A$ and
  an equivalence $C \cong N(A^\circ)$.
\end{enumerate}

Here $A^{\circ}$ is the underlying category of bifibrant objects.
\end{theorem}

\begin{remark}(HTT A.3.7.7)

Let $A$ and $B$ be combinatorial simplicial model categories. Then the
underlying $\infty$-categories $N(A^\circ)$ and $N(B^\circ)$ are
equivalent iff $A$ and $B$ can be joined by a chain of simplicial
Quillen equivalences. 
\end{remark}

\begin{theorem}(HTT 5.2.4.6)
Let $A$ and $B$ be simplicial model categories, and let 
\[
\xymatrix{A \ar@<.5ex>[r]^F & \ar@<.5ex>[l]^G B}
\]

be a simplicial Quillen adjunction. This descends to an adjunction on
the underlying $\infty$-categories.
\end{theorem}



\end{document}