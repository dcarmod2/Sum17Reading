\documentclass[letterpaper]{article}
\usepackage{fancyhdr}
\usepackage[margin=1in]{geometry}
\usepackage{amsmath,amsthm,amssymb,amsfonts,amsopn,mathrsfs,mathtools}
%\usepackage{parskip}
\usepackage{tikz}
\usetikzlibrary{arrows}
\usepackage[all]{xy}
\usepackage{pdfpages}
\usepackage[compact]{titlesec}
\usepackage[normalem]{ulem}
\usepackage{cite}
\usepackage{hyperref}
\usepackage{color}
\usepackage{enumitem}
\usepackage{tabulary}
\usepackage{titlesec}
\usepackage{hyperref}

\hypersetup{colorlinks=true,linktoc=all,linkcolor=blue}

\newtheorem{lemma}{Lemma}
\newtheorem{theorem}{Theorem}
\newtheorem{proposition}[lemma]{Proposition}
\theoremstyle{definition}
\newtheorem{corollary}[lemma]{Corollary}
\newtheorem{example}[lemma]{Example}
\newtheorem{definition}[lemma]{Definition}

\newtheorem{remark}[lemma]{Remark}
\newtheorem{question}{Question}
\newtheorem{conjecture}{Conjecture}
\newtheorem{construction}{Construction}



\newcommand{\Z}{\mathbb Z}

\newcommand{\mbb}{\mathbb}
\newcommand{\mc}{\mathcal}


\DeclareMathOperator{\Spec}{Spec}
\DeclareMathOperator*{\holim}{holim}
\DeclareMathOperator*{\hocolim}{hocolim}
\DeclareMathOperator*{\colim}{colim}
\DeclareMathOperator*{\im}{im}
\DeclareMathOperator*{\coker}{coker}

\setcounter{secnumdepth}{4}
\titleformat{\paragraph}
{\normalfont\normalsize\bfseries}{\theparagraph}{1em}{}
\titlespacing*{\paragraph}
{0pt}{3.25ex plus 1ex minus .2ex}{1.5ex plus .2ex}





 \setlength{\headheight}{0pt}
  \lhead{Daniel, Brian, Tsutomu}\chead{UnivKThy Notes}\rhead{Summer 2017}
  \lfoot{}\cfoot{\thepage}\rfoot{}
  \pagestyle{fancy}


\begin{document}

\begin{remark}
Notation: $K(-)$ is the connective $K$-theory spectrum, and $\mbb
K(-)$ is the non-connective $K$-theory spectrum. 
\end{remark}

\begin{theorem}
There's a LES of $K$-groups
\[
\xymatrix{ \cdots \ar[r] & K_{n+1}(\mbb Q) \ar[r] & \bigoplus_{p} K_n(\mbb F_p) \ar[r] &
  K_n(\Z) \ar[r] & K_n(\mbb Q) \ar[r] & \dots}
\]

Thanks to hard work about $K$ theory of fields, gives lots of
information about $K_*(\Z)$.
\end{theorem}

This theorem is (non-trivially) a special case of:

\begin{theorem} (Quillen)
Given $B \subset A$ a Serre subcategory of a small abelian category,
then we have a cofiber sequence of spectra
\[
K(B) \rightarrow K(A) \rightarrow K(A/B).
\]

Note that in particular, $K_0(A) \twoheadrightarrow K_0(A/B)
\rightarrow 0$.
\end{theorem}

\begin{remark}
This is a premier computational tool. However, not all $K$-groups are
equivalent to $K$-groups of some abelian category. Notably, $K_*$ of a
singular scheme, $K_*$ of a ring spectrum, etc. 
\end{remark}

\begin{remark}
If $A \rightarrow B \rightarrow C$ is an exact sequence in
$Cat_{\infty}^{perf}$, it's not true in general that
\[
K_0(B) \rightarrow K_0(C) \rightarrow 0
\]
is exact. This is the first obstruction to a localization sequence. 
\end{remark}

\begin{question}
Is there an easy counterexample? I think possibly a singular cubic
will work. 
\end{question}

Fix: Negative $K$-theory.

\begin{theorem}
If $A \rightarrow B \rightarrow C$ is an exact sequence in
$Cat_{\infty}^{ex}$, then 
\[
\mbb K(A) \rightarrow \mbb K(B) \rightarrow \mbb K(C)
\]
is a cofiber sequence.
\end{theorem}

\begin{construction}
Idea: Want to find a $C'$ s.t. $K(C') \cong *$ and a map $C
\rightarrow C'$. Expect exact sequence
\[
\xymatrix{K_0(C') \ar[r] & K_0(C'/C) \ar[r] & K_{-1}(C) \ar[r] & K_{-1}(C')}
\]
but since $K_*(C') = 0$, just define $K_{-1}(C) = K_0(C'/C)$. Should
be functorial: $C' = \mc F(C)$. Let $\Sigma C = cofib(C \rightarrow
F(C))$, then $K_{-n}(C) = K_0(\Sigma^{(n)}C)$.
\end{construction}

\begin{definition}
Say $C$ is \emph{flasque} if there are exact functors $F_1,F_2 : C
\rightarrow C$ and equivalence
\[
id \oplus F_1 \cong F_2,
\]
and $(F_1)_* = (F_2)_* : K_*(C) \rightarrow K_*(C)$. 


\end{definition}

Then $id +
(F_1)_* : K_*(C) \rightarrow K_*(C) \implies K_*(C) = 0$.

\begin{example}
\begin{enumerate}
\item $Ind_\kappa(C)$, $\kappa > \omega$.
\item $F = (x \mapsto \oplus_{\mbb N} x)$. Then $id \oplus F \cong F$,
  the Eilenberg-swindle, so $Ind_\kappa(\mc C)$ is $\kappa$-acyclic. 
\item A ring $R$ is flasque if there's an $R$-bimodule $M$ which is
  f.g. projective as a right $R$-mod, and there's a bimodule
  isomorphism $R\oplus M \cong M$. Then $Mod_R, Proj_R$ are flasque. 
\item $S$ any ring, $C(R) \subseteq End_S(S^\infty)$ row-finite
  column-finite infinite matrices (the cone ring). This is flasque. 
\item The suspension ring $\Sigma S = C(S)/M(S)$, where $M(S)$ denotes
  the ring of finite matrices. Then $K_{-n}(S) = K_0(\Sigma^{(n)} S)$.
\end{enumerate}
\end{example}

\begin{remark}
This construction works just fine for connective ring spectrum. 
\end{remark}

\begin{construction}
Model for $K$-theory of connective rings: 

Can define

\[
\xymatrix{GL_n(R) \ar[r]\ar[d] & M_n(R) \ar[d] \\ GL_n(\pi_0R) \ar[r]
  & M_n(\pi_0R)}
\]

Fact: $K_0(R) \cong K_0(\pi_0 R)$. Here the $K$-theory of a ring
spectrum is the $K$-theory of the category of compact projective $R$-mods.
Define $K_*(R) := \pi_*[K_0(\pi_0R) \times BGL(R)^+]$.

Have a map $K_*(R) \rightarrow K_*(\pi_0(R))$. Compare the ``ring
suspensions'':

have $\pi_0(\Sigma_{ring} R) = \Sigma_{ring} \pi_0R$, so 

\[
\xymatrix{K_{-1}(R) \ar[r]\ar[d] & K_0(\Sigma_{ring}R)
\ar[d]\\K_{-1}(\pi_0R) \ar[r] &K_0(\pi_0\Sigma_{ring}R)}
\]

Upshot: $K_{-n}(R) = K_{-n}(\pi_0R)$ for $R$ connective. 
\end{construction}

\begin{definition}
$C \in Cat_{\infty}^{ex}$. Define $\mc F_k(C) = Ind_\kappa(C)$,
$\Sigma_\kappa C = \mc F_\kappa (C)/C$, $K_{-n}(C) =
K_0(\Sigma_\kappa^{(n)} C)$, $\mbb K(C) = \colim_n \Omega^n
K(\Sigma^{(n)}_\kappa C)$. This definition doesn't make any
multiplicative properties apparent. 
\end{definition}

What is known:

\begin{enumerate}
\item $K_{-n}(\text{noetherian regular ring/scheme}) = 0$
\item $K_{-1}(\text{henselian ring}) = 0$ (hard-Drinfeld)
\item Weibel's conjecture: $X$: noetherian scheme of dimension =
  $d$. Then $K_{-n}(X)$ are zero if $n > d$. This is a theorem now ($d
  = 1$, Bass), ($d = 2$, Weibel), ($X$ variety over field of char 0,
  Haesemeyer-Cortinas-Schlicting-Weibel), ($X/F$ char$(F) > 0$
  assuming res of singularites, Geisser-Hesselholt/Krishna), (whole
  thing, Kerz-Strunk-Tamme Nov '16).
\item Schlicting's conjecture: $K_{-n}(A) = 0$ if $A$ (small)
  abelian. Still open in general. True if $A$ is noetherian. Also true
  for $n = 1$ for any small abelian category. 
\item If $E \in Cat_\infty^{ex}$ has a bounded $t$-structure,
  $K_{-1}(E) = 0$. If ... with $E^{heart}$ noetherian, $K_{-n}(E) =
  0$ for all $n \geq 1$. 
\end{enumerate}

Idea of $K_{-1}(A) = 0$. 
\[
\xymatrix{D^b(A) \ar[r]\ar[d] & D^{-}(A) \ar[d] \\ D^+A \ar[r] & D(A)}
\]

induces pushout square on $K$-theory. $D^+A$ and $D^{-}A$ are
idempotent complete and $K$-theory acyclic. $K_*(A) = K_*(D^b(A))$,
$K_0(D(A)) = K_{-1}(A)$. 
\end{document}